\section{Background}
\label{SecBackground}

Empirical studies on refactoring rely on means to identify refactoring activity. Thus, different techniques have been proposed and employed for this task.
For example, Murphy-Hill~et~al.~\cite{MurphyHill2012} collected refactoring usage data using a plug-in that monitors user actions in the Eclipse IDE, including calls to refactoring commands.
%Negara~et~al.~\cite{negara2013} also used the strategy of instrumenting the IDE to infer refactorings from fine-grained code edits.
Negara~et~al.~\cite{negara2013} describe a tool, called CodingTracker, to infer refactorings from fine-grained code edits. They use this tool to study refactorings performed by 23 developers working in their IDEs during a total of 1,520 hours. The tool achieved a precision of 99.3\% when evaluated with the automated Eclipse refactorings performed by the study participants.
On a sample of both manual and automated refactorings, CodingTracker achieved a precision of 93\% and a recall of 100\%.
However, CodingTracker requires the installation of a refactoring inference plugin in IDEs.

Other studies use metadata from version control systems to identify refactoring changes. For example, Ratzinger~et~al.~\cite{ratzinger2008relation} search for a predefined set of terms in commit messages to classify them as refactoring changes. In specific scenarios, a branch may be created exclusively to refactor the code, as reported by Kim et al.~\cite{kim-tse-2014}.
Another strategy is employed by Soares et al.~\cite{soares2010making}. They propose an approach that identifies behavior-preserving changes by automatically generating and running test-cases. While their approach is intended to guarantee the correct behavior of a system after refactoring, it may also be employed to classify commits as behavior-preserving.
Moreover, many existing approaches are based on static analysis.
This is the case of the approach proposed by Demeyer et al.~\cite{demeyer2000finding}, which finds refactored elements by observing changes in code metrics.

Static analysis is also frequently used to find differences in the source code by comparing two revisions~\cite{dig2006automated, weissgerber2006identifying, tsantalis_empiricalstudy,prete2010template,Kim:2010:RefFinder,msr2017,tsantalis2018rminer}.
Approaches based on comparing source code differences have the advantage of beeing able to identify refactoring operations applied in version histories.
As RefDiff is one of these approaches, it can be directly compared with others within this category. In the next sections, we discuss RefDiff~1.0 and three other approaches.

\subsection{RefDiff~1.0}

The original version of RefDiff~\cite{msr2017}, which we will denote as RefDiff~1.0 throughout this paper, employs a combination of heuristics based on static analysis and code similarity to detect 13 well-known refactoring types.
One of its distinguishing characteristic is the use of the classical TF-IDF similarity measure from information retrieval to compute code similarity.
In our previous work, we evaluated RefDiff~1.0 using an oracle of 448 refactoring operations, distributed across seven Java projects.
We built this oracle by deliberately applying refactorings in software repositories in a controlled manner.
Although this strategy poses the risk of creating an artificial dataset, this way we assured this oracle was complete and could be used to compute both precision and recall.
We compared our tool with three existing approaches, namely Refactoring Miner~\cite{tsantalis_empiricalstudy}, Refactoring Crawler~\cite{dig2006automated}, and Ref-Finder~\cite{Kim:2010:RefFinder}.
Our approach achieved precision of 100\% and recall of 88\%, surpassing the three tools subjected to the comparison.


\subsection{Refactoring Miner/RMiner}

Refactoring Miner is an approach originally introduced by Tsantalis~et~al.~\cite{tsantalis_empiricalstudy}, capable of identifying 14 high-level refactoring types: \emph{Rename Package/Class/Method}, \emph{Move Class/Method/Field}, \emph{Pull Up Method/Field}, \emph{Push Down Method/Field}, \emph{Extract Method}, \emph{Inline Method}, and \emph{Extract Superclass/Interface}.
In its original version, Refactoring Miner employs a lightweight algorithm, similar to the UMLDiff proposed by Xing and Stroulia~\cite{Xing:2005}, for differencing object-oriented models, inferring the set of classes, methods, and fields added, deleted or moved between two code revisions. 
Refactoring Miner was employed and evaluated in empirical studies on refactoring along its evolution.
In the first study, using the version histories of JUnit, HTTPCore, and HTTPClient, Tsantalis~et~al.~\cite{tsantalis_empiricalstudy} reported 8 false positives for the \emph{Extract Method} refactoring (96.4\% precision) and 4 false positives for the \emph{Rename Class} refactoring (97.6\% precision). No false positives were reported for the remaining refactorings.
In a second study that mined refactorings in 285 GitHub hosted Java repositories~\cite{fse2016-why-we-refactor}, we found found 1,030 false positives out of 2,441 refactorings (63\% precision). However, we also evaluated Refactoring Miner using as a benchmark the dataset reported by Chaparro~et~al.~\cite{Chaparro:2014}, in which it achieved 93\% precision and 98\% recall.

In a recent study, Tsantalis~et~al.~\cite{tsantalis2018rminer} proposed a major evolution of its tool, now named as RMiner (version 1.0).
RMiner relies on an AST-based statement matching algorithm and a set of detection rules that cover 15 representative refactoring types. 
Its statement matching algorithm employes two techniques to be resilient to code restructuring during refactoring: abstraction, which deals with changes in statements' AST type due to refactorings, and argumentization, which deals with changes in sub-expressions within statements due to parameterization.
To evaluate RMiner, the authors created a dataset with 3,188 real refactorings instances from 185 open-source projects. Using this oracle, the authors found that RMiner has a precision of 98\% and recall of 87\%, which was the best result so far, surpassing RefDiff 1.0, the previous state-of-the-art, which achieved precision of 75.7\% and recall of 85.8\% in this dataset.


%that was later extend by Silva~et~al.~\cite{fse2016-why-we-refactor} to mine refactorings in large scale in git repositories. This tool is capable of identifying 14 high-level refactoring types: \emph{Rename Package/Class/Method}, \emph{Move Class/Method/Field}, \emph{Pull Up Method/Field}, \emph{Push Down Method/Field}, \emph{Extract Method}, \emph{Inline Method}, and \emph{Extract Superclass/Interface}.

%Refactoring Miner runs a lightweight algorithm, similar to the UMLDiff proposed by Xing and Stroulia~\cite{Xing:2005}, for differencing object-oriented models, inferring the set of classes, methods, and fields added, deleted or moved between two code revisions. 
%First, the algorithm matches code entities in a top-down order (starting from the classes and going to the methods and fields) looking for exact matches on their names and signatures (in the case of methods).
%Next, the removed/added elements between the two models are matched based only on the equality of their names in order to find changes in the signatures of fields and methods.
%Third, the removed/added classes are matched based on the similarity of their members at signature level.
%Finally, a set of rules enforcing structural constraints is applied to identify specific types of refactorings.



\subsection{Refactoring Crawler}

Refactoring Crawler, proposed by Dig~et~al.~\cite{dig2006automated}, is an approach capable of finding seven high-level refactoring types: \emph{Rename Package/Class/Method}, \emph{Pull Up Method}, \emph{Push Down Method}, \emph{Move Method}, and \emph{Change Method Signature}.
It uses a combination of syntactic analysis to detect refactoring candidates and a reference graph analysis to refine the results.

First, Refactoring Crawler analyzes the abstract syntax tree of a program and produces a tree, in which each node represents a source code entity (package, class, method, or field).
Then, it employs a technique known as \emph{shingles encoding} to find 
similar pairs of entities, which are candidates for refactorings.
Shingles are representations for strings with the following property: if a string changes slightly, then its shingles also change slightly.
In a second phase, Refactoring Crawler applies specific strategies for detecting each refactoring type, and computes a more costly metric that determines the similarity of references between code entities in two versions of the system. For example, two methods are similar if the sets of methods that call them are similar, and the sets of methods they call are also similar.
The strategies to detect refactorings are repeated in a loop until no new refactorings are found. Therefore, the detection of a refactoring, such as a rename, may change the reference graph and enable the detection of new refactorings.

The authors evaluated Refactoring Crawler comparing pairs of releases of three open-source software components: Eclipse UI, Struts, and JHotDraw. Such components were chosen because they provided detailed release notes describing API changes. The authors relied on such information and on manual inspection to build an oracle
%of known refactorings in those releases
containing 131 refactorings.
The reported results are: Eclipse UI (90\% precision and 86\% recall), Struts (100\% precision and 86\% recall), and JHotDraw (100\% precision and 100\% recall).
%However, in our previous study~\cite{msr2017}, Refactoring Crawler achieved only 41.9\% of precision and 35.6\% of recall.

\subsection{Ref-Finder}

Ref-Finder, proposed by Prete~et~al.~\cite{prete2010template,Kim:2010:RefFinder}, is an approach based on logic programming capable of identifying 63 refactoring types from the Fowler's catalog\cite{Fowler:1999}.
The authors express each refactoring type by defining structural constraints, before and after applying a refactoring to a program, in terms of template logic rules.

First, Ref-Finder traverses the abstract syntax tree of a program and extracts facts about code elements, structural dependencies, and the content of code elements, to represent the program in terms of a database of logic facts. Then, it uses a logic programming engine to infer concrete refactoring instances, by creating a logic query based on the constraints defined for each refactoring type.
The definition of refactoring types also consider ordering dependencies among them. This way, lower-level refactorings may be queried to identify higher-level, composite refactorings.
The detection of some types of refactoring requires a special logic predicate that indicates that the similarity between two methods is above a threshold. For this purpose, the authors implemented a block-level clone detection technique, which removes any beginning and trailing parenthesis, escape characters, white spaces, and return keywords and computes word-level similarity between the two texts using the longest common sub-sequence algorithm.

The authors evaluated Ref-Finder in two case studies.
In the first one, they used code examples from the Fowler's catalog to create instances of the 63 refactoring types. The authors reported 93.7\% recall and 97.0\% precision for this first study.
In the second study, the authors used three open-source projects: Carol, jEdit, and Columba. In this case, Ref-Finder was executed in randomly selected pairs of versions. From the 774 refactoring instances found, the authors manually inspected a sample of 344 instances and found that 254 were correct (73.8\% precision).
%However, in a study by Soares~et~al.~\cite{Soares:2013} using a set of randomly select revisions of JHotDraw and Apache Common Collections containing 81 refactoring instances in total, Ref-Finder achieved 35\% of precision and 24\% of recall.
%Moreover, in our previous study~\cite{msr2017}, it also achieved 26.4\% of precision and 64.2\% of recall.

It is worth noting that Ref-Finder and Refactoring Crawler require a full build of the program under analysis.
Therefore, their usage is not recommended when mining refactorings from version histories in the large.
In this case, it might be a challenge to build each release, due to missing external dependencies, for example.
For that reason, our evaluation (Section~\ref{sec:eval:java}) focus on comparing RefDiff with RMiner.

