\section{Challenges and limitations}
\label{sec:challenges}


\textbf{Low-level refactorings:} RefDiff does not detect local refactorings, such as rename, extract or inline a local variable, because the syntactical structure of the source code within a CST node is not represented.
While it is theoretically possible to extend the CST to include finer-grained code elements such as statements, local variables, and others, this would also make it harder to port RefDiff to other programming languages.


\noindent\textbf{Generating call graphs:} In our modular architecture, the generation of the CST, which includes information from a call graph and a type hierarchy graph, is delegated to a language-specific plugin.
For languages such as Java, there are reliable parsers and static analyzers that aid in this task (e.g., Eclipse JDT).
However, we acknowledge that generating precise call graphs for untyped languages, such as JavaScript, might be a challenging problem.
Nevertheless, we provided evidences that our approach works well even when the information encoded in the CST is not completely precise.
For example, in our JavaScript implementation---which contains only 689 lines of code in total---we used a simple strategy in which we assume a node $n_1$ uses $n_2$ if $n_1$ contains a function call with the same identifier as $n_2$ and both are defined in the same file.
However, to detect a \emph{Extract} relationship between $n_1$ and $n_2$, we need two other conditions: (i) $n_2$ should be a new method and (ii) the body of $n_2$ should be similar to the code removed from $n_1$ between revisions.
In other words, an incorrect edge in the call graph only leads to an incorrect \emph{Extract} relationship in the unlikely scenario in which a function $n_2$ is introduced, the content of such function is similar to code removed from $n_1$ and $n_1$ calls a function with the same identifier of $n_2$ after the change, but that function is not actually $n_2$.
A similar reasoning applies to \emph{Inline} relationships, which also depends on information from call graphs.
In summary, although generating precise call graphs is non trivial for untyped languages, we argue that it is not needed in practice to achieve acceptable  precision, specially in the light of the results of our evaluation using JavaScript systems (91\% of precision).

\noindent\textbf{JavaScript class syntax:} Our JavaScript implementation only considers classes defined with the new ES6 syntax, i.e., classes emulated by  functions definitions and prototype-based inheritance are just treated like regular functions when generating the CST.

\noindent\textbf{Field-related refactorings:}

As our refactoring detection algorithm is centered around code similarity and fields do not have a body, we did not implement the detection of \emph{Move/Pull Up/Push Down Field} in RefDiff 2.0.
Unrestricted detection of \emph{Move Field} based solely on fields' types and names is prone to find many false positives.
However, we plan to add support for field-related refactorings in future work by using stricter detection rules, similarly to RMiner (e.g., requiring a dependency between their source and destination classes).

%Without resorting to code similarity, we need other means to avoid detecting many false positives. In RefDiff 1.0 we tried to solve that problem by using the statements that used the field in our similarity index. However, such strategy did not yield good results in practice.
