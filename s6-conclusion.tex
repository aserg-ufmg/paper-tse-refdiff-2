\section{Conclusion}
\label{SecConclusion}



%In this paper, we proposed RefDiff 2.0, an extension a refactoring detection approach that is, from the best of our knowledge, 
To the best of our knowledge, RefDiff 2.0 is the first refactoring detection approach that supports multiple programming languages.
%approach designed from the ground up to support multiple programming languages.
We made this possible with two main design decisions. First, our refactoring detection algorithm relies only on information encoded in CSTs, a data structure that represents the source code but abstracts the specificities of each programming language.
Second, we compute code similarity at the level of the tokenized source code, using techniques from information retrieval.
In summary, RefDiff is loosely coupled to the syntax of the target programming language, which makes it easier to extend it to other languages.

Our evaluation using a dataset of real refactorings in Java showed that RefDiff's precision is 96.4\% and recall is 80.4\%.
Although we were not able to surpass RMiner's precision of 98.8\%, we argue that we achieved satisfactory results for a language-neutral approach.
In one hand, specialized tools can use more advanced techniques to improve refactoring detection.
For example, RMiner relies on a statement matching algorithm and applies syntax-aware replacement of AST nodes to make it more resilient to code changes overlapped with refactorings.
On the other hand, the higher the coupling with the syntax of a particular language, the harder it becomes to port the approach to other programming languages.

%This is a tradeoff
Last, our evaluation in JavasScript and C also showed promising results. RefDiff's precision and recall are respectively 91\% and 88\% for JavaScript, and 88\% and 91\% for C.
These results show the viability our approach for languages other than Java.
Thus, we claim that RefDiff~2.0 can pave the way for important advances in refactoring studies in JavaScript, C, and other languages in the future.
Moreover, it can be employed in practical tasks, such as improving diff visualization, automatically documenting refactorings in commits, keeping track of the history of refactored code elements, and others.
