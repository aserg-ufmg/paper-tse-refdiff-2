\section{Proposed Approach}

Our approach consists in two phases: Source Code Analysis and Relationship Analysis.
In the first phase, Source Code Analysis, we take as input two revisions of a system, $v_1$ and $v_2$, and build two models that represent their source code.
Both models have the form of a tree, in which each node corresponds to a code element (classes, functions, etc.).
In the second phase, Relationship Analysis, we compute the set $R$, which contains triples of the form $(n_1, n_2, t)$, where $n_1$ is a code element from revision $v_1$, $n_2$ is a code element from revision $v_2$ and $t$ is a relationship type.
Such relationships may denote a high-level refactoring operation (move, rename, extract, etc.) or an exact correspondence between the code elements.
For example, consider the diff between two revisions of a system depicted in Figure~\ref{FigDiff1}.
Among other changes, the class \codeinl{Calculator}, declared in revision~1, is renamed to \codeinl{FpCalculator} in revision~2. This would correspond to a relationship of the type \emph{Rename} between them.
In the next sections we describe in details each phase of our approach.

\begin{figure*}[htb]
\centering
\includegraphics[width=0.9\textwidth]{img/diff1.pdf}
\caption{Illustrative diff between two revisions of a system}
\label{FigDiff1}
\end{figure*}


\subsection{Phase 1: Source Code Analysis}

The goal of this phase is compute a language-independent model that represents the source code of the system, which we denote from now on as \emph{Code Structure Tree} (CST). The CST is a tree-like structure that resembles an \emph{Abstract Syntax Tree} (AST), however, in this representation we are interested in coarse-grained code elements, that encompass a code region and may be referred by an identifier in other parts of the system.

To construct the CST, we need to parse the source code, generate the AST for the target programming language, and extract the necessary information.
Thus, the decision of which types of AST nodes becomes CST nodes depends on language.
For example, in Java we represent classes, enums, interfaces, and methods as CST nodes.
In contrast, local variables are not represented.
Nevertheless, it is important to note that the granularity of the CST nodes determines the granularity of the relationships we are able to find, e.g., we can only find relationships between methods if we represent methods in the CST.
Table~\ref{TabCstNodes} lists the types of AST nodes that are represented in the CST for each programming language we support.

\begin{table}[htbp]
\renewcommand{\arraystretch}{1.2}
\caption{AST nodes that are represented in the CST}
\label{TabCstNodes}
\centering
\begin{tabular}{@{}ll@{}}
\toprule
Language & Node types \\
\midrule
Java & class, enum, interface, and method \\
C & file and function \\
JavaScript & file, class, and function \\
\bottomrule
\end{tabular}
\end{table}

\begin{figure}[htb]
\centering
\includegraphics[width=1.0\linewidth]{img/cstDiff1.pdf}
\caption{CST of both revisions of the example system from Figure~\ref{FigDiff1}}
\label{FigJavaToCst}
\end{figure}

Figure~\ref{FigJavaToCst} exemplifies the transformation of the example system from Figure~\ref{FigDiff1} into a corresponding CST.
In revision~1, the class \codeinl{Main} is declared with a single method \codeinl{main} and the class \codeinl{Calculator} contains two methods: \codeinl{sum} and \codeinl{min}.
Note that all those classes and methods become nodes in the CST for revision~1, preserving the same nesting structure of the source code. Analogously, the figure also depicts the CST for revision 2, which contains seven nodes in total (two classes and five methods).

Besides the representation of the code elements, the CST also holds a simplified call graph and  a type hierarchy graph of the nodes within the CST, that is, there are edges to represent whether a certain node $n_1$ calls $n_2$, or $n_1$ is a subtype of $n_2$. The first information is necessary to find \emph{Extract} and \emph{Inline} relationships between code elements, while the second is used to find inheritance related relationships such as \emph{Pull Up} and \emph{Push Down}.

Moreover, along with each node of the CST, we store the following information:
\begin{description}
    \item[Identifier] \hfill \\
    An identifier of the code element in its declared scope. 
    The identifier is usually the name of the code element, but it may also contain additional information to avoid ambiguities.
    For example, the identifier of the class \codeinl{Calculator} from Figure~\ref{FigJavaToCst} is simply its name, but the identifier of the method \codeinl{sum} is \codeinl{sum(double,double)} because there could be an overloaded method with a different signature.
    
    \item[Namespace] \hfill \\
    An optional prefix that, along with the identifier, globally identifies the code element. 
    This information only applies to root nodes and corresponds to the package or folder that the element is contained. For example, the namespace of the class \codeinl{Calculator} from Figure~\ref{FigJavaToCst} is \codeinl{my.calc.}.
    
    \item[Node type] \hfill \\
    A string that identifies the node type in the target language (class, function, method, etc.).
    
    \item[Parameters list]  \hfill \\
    An optional list of the name of the parameters, in the case the node corresponds to a method or function.
    
    \item[Tokenized source code]  \hfill \\
    The source code of the code element in the form of a list of tokens. 
    This information is necessary to compute the similarity between code elements, as explained in Section~\ref{SecCodeSim}.
    
    \item[Tokenized source code of the body]  \hfill \\
    The source code of the body of the code element in the form of a list of tokens. 
    This information is optional, as not every node possess a body (e.g., abstract members).
    This information is also necessary to compute the similarity between code elements in the special cases of \textit{Extract} and \textit{Inline} relationships, as explained in Section~\ref{SecSimX}.
    
\end{description}

It is worth noting that we generate the CST only for source files that have been added, removed, or modified between revisions. 
Such information can be efficiently obtained from version control systems, without the need to read the content of all files within the repository.
This way, we avoid a costly operation that might compromise the scalability of our approach, as large repositories contains thousands of source files, but only a small fraction of them change between subsequent revisions.

Although the construction of the CST is a language-specific process, from this point on, the approach is language-independent and relies only on the information encoded in the CST.
This way, one is able to extend our approach to work with different programming languages only by implementing the \emph{Source Code Analysis} module.
To demonstrate that capability, we provide implementations for three programming languages: Java, C, and JavaScript.


\subsection{Phase 2: Relationship Analysis}


\begin{table*}[htbp]
\renewcommand{\arraystretch}{1.3}
\caption{Relationship types and the conditions to find them}
\label{TabRelationshipTypes}
\centering

\begin{tabular}{@{}lll@{}}
\toprule
Relationship type & \multicolumn{2}{l}{Conditions} \\
\midrule
& \multicolumn{2}{l}{$(n_1, n_2) \in N^- \times N^+$, such that:}\\
Same & & $\rdtype(n_1) = \rdtype(n_2) \land \rdsig(n_1) = \rdsig(n_2) \land \rdparent(n_1)' = \rdparent(n_2)$ \\
Convert Type & & $\rdtype(n_1) \neq \rdtype(n_2) \land \rdsig(n_1) = \rdsig(n_2) \land \rdparent(n_1)' = \rdparent(n_2)$ \\
Pull Up & & $\rdtype(n_1) = \rdtype(n_2) \land \rdsig(n_1) = \rdsig(n_2) \land \rdsub(\rdparent(n_1)', \rdparent(n_2))$ \\
Push Down & & $\rdtype(n_1) = \rdtype(n_2) \land \rdsig(n_1) = \rdsig(n_2) \land \rdsub(\rdparent(n_2), \rdparent(n_1)')$ \\
Change Signature & & $\rdtype(n_1) = \rdtype(n_2) \land \rdsig(n_1) \neq \rdsig(n_2) \land \rdname(n_1) = \rdname(n_2) \land \rdparent(n_1)' = \rdparent(n_2) \land \rdsim(n_1, n_2) > \tau$ \\
Move & & $\rdtype(n_1) = \rdtype(n_2) \land \rdname(n_1) = \rdname(n_2) \land \rdparent(n_1)' \neq \rdparent(n_2) \land \rdsim(n_1, n_2) > \tau$ \\
Rename & & $\rdtype(n_1) = \rdtype(n_2) \land \rdname(n_1) \neq \rdname(n_2) \land \rdparent(n_1)' = \rdparent(n_2) \land \rdsim(n_1, n_2) > \tau$ \\
Move and Rename & & $\rdtype(n_1) = \rdtype(n_2) \land \rdname(n_1) \neq \rdname(n_2) \land \rdparent(n_1)' \neq \rdparent(n_2) \land \rdsim(n_1, n_2) > \tau$ \\
\addlinespace
& \multicolumn{2}{l}{$(n_1, n_2) \in M_1 \times N^+$, such that:}\\
Extract Supertype & & $\exists (n_3, n_4, \mathit{PullUp}) \in R\, (n_1 = \rdparent(n_3) \land n_2 = \rdparent(n_4))$ \\
Extract & & $\rduses(n_1', n_2) \land \rdparent(n_1)' = \rdparent(n_2) \land \rdsimx(n_2, n_1) > \tau$ \\
Extract and Move & & $\rduses(n_1', n_2) \land \rdparent(n_1)' \neq \rdparent(n_2) \land \rdsimx(n_2, n_1) > \tau$ \\
\addlinespace
& \multicolumn{2}{l}{$(n_1, n_2) \in N^- \times M_2$, such that:}\\
Inline & & $\rduses(n_1, n_2') \land \rdsimx(n_1, n_2) > \tau$ \\
\bottomrule
\end{tabular}

\vspace{1em}
\begin{tabular}{@{}lll@{}}
\midrule
\multicolumn{3}{c}{\textbf{Definitions}}\\
\begin{tabular}{@{}ll@{}}
$M_1$ & the set of nodes from $N_1$ that matches with another node from $N_2$\\
$M_2$ & the set of nodes from $N_2$ that matches with another node from $N_1$\\
$N^-$ & the set of unmatched nodes from $N_1$ ($N_1 \setminus M_1$)\\
$N^+$ & the set of unmatched nodes from $N_2$ ($N_2 \setminus M_2$)\\
$n'$ & the code element that matches with $n$ in the other revision\\
$\rdparent(n)$ & parent of a node $n$ (it may be a namespace or another CST node)\\
$\rdns(n)$ & namespace of the code element $n$\\
\end{tabular}
& &
\begin{tabular}{@{}ll@{}}
$\rdname(n)$ & simple name of the code element $n$\\
$\rdsig(n)$ & identifier of the code element $n$\\
$\rdtype(n)$ & node type of the code element $n$\\
$\rdsub(n_1, n_2)$ & $n_1$ is subtype of $n_2$\\
$\rduses(n_1, n_2)$ & $n_1$ uses $n_2$\\
$\rdsim(n_1, n_2)$ & similarity between $n_1$ and $n_2$\\
$\rdsimx(n_1, n_2)$ & extract similarity between $n_1$ and $n_2$\\
\end{tabular}
\\
\midrule
\end{tabular}

\end{table*}



This phase takes as input the CST's of revisions $v_1$ and $v_2$ and outputs the set of relationships $R$. Let $N_1$ and $N_2$ be the sets of code elements from the CST's of $v_1$ and $v_2$ respectively, each relationship $r \in R$ is a triple $(n_1, n_2, t)$, where $n_1 \in N_1$, $n_2 \in N_2$, and $t$ is a relationship type. The types of relationships are listed in the first column of Table~\ref{TabRelationshipTypes}, and can be subdivided into two groups:
\begin{itemize}
\item \textbf{Matching relationships}, which include \textit{Same}, \textit{Convert Type}, \textit{Pull Up}, \textit{Push Down}, \textit{Change Signature}, \textit{Move}, \textit{Rename}, and \textit{Move and Rename}.
Those relationships indicates that the node $n_1$ corresponds to $n_2$ in the subsequent revision.
We say that a node $n_1$ matches with $n_2$ if exists a relationship $(n_1, n_2, t) \in R$ such that $t$ is a matching relationship.

\item \textbf{Non-matching relationships}, which include \textit{Extract Supertype}, \textit{Extract}, \textit{Extract and Move}, and \textit{Inline}.
Those relationships don't indicate a matching, but rather that either node $n_1$ is decomposed to create $n_2$, or node $n_1$ is incorporated into $n_2$.
\end{itemize}


\subsubsection{General procedure to find relationships}

\begin{figure}[htbp]
\begin{algorithmic}[1]
\Procedure{FindRelationships}{$t_1,t_2$}
\State $R \gets \emptyset$
\State $M \gets \emptyset$
\State $\textsc{findMatchingsById}(t_1,t_2)$
\State $\textsc{findMatchingsBySim}$
\State $\textsc{resolveMatchings}$
\State $\textsc{findNonMatchingRel}$
\State \Return $R$
\\
\Procedure{findMatchingsById}{$p_1,p_2$}
\ForEach{$(n_1, n_2) \in \rdchildren(p_1) \times \rdchildren(p_2)$}
  \If {$\rdsig(n_1) = \rdsig(n_2) \land \rdns(n_1) = \rdns(n_2)$}
    \State $\textsc{addMatch}(n_1, n_2)$
  \EndIf
\EndFor
\EndProcedure
\\
\Procedure{findMatchingsBySim}{}
\State $M' \gets N^- \times N^+$
\While{$M' \neq \emptyset$}
  \State $(n_1, n_2) \gets \argmin_{m \in M'} \rdrank(m)$
  \State $M' \gets M' \setminus \{(n_1, n_2)\}$
  \If {$\mathit{findMatchRel}(n_1, n_2) \neq \emptyset$}
    \State $\textsc{addMatch}(n_1, n_2)$
  \EndIf
\EndWhile
\EndProcedure
\\
\Procedure{resolveMatchings}{}
\ForEach{$(n_1, n_2) \in M$}
  \State $R \gets R \cup \mathit{findMatchRel}(n_1, n_2)$
\EndFor
\EndProcedure
\\
\Procedure{findNonMatchingRel}{}
\ForEach{$(n_1, n_2) \in N_1^= \times N^+$}
  \State $R \gets R \cup \mathit{findExtractSupertype}(n_1, n_2)$
  \State $R \gets R \cup \mathit{findExtract}(n_1, n_2)$
  \State $R \gets R \cup \mathit{findExtractMove}(n_1, n_2)$
\EndFor
\ForEach{$(n_1, n_2) \in N^- \times N_2^=$}
  \State $R \gets R \cup \mathit{findInline}(n_1, n_2)$
\EndFor
\EndProcedure
\\
\Procedure{addMatch}{$n_1, n_2$}
\If {$n_1 \in N^- \land n_2 \in N^+$}
  \State $M \gets M \cup \{(n_1, n_2)\}$
  \State $\textsc{findMatchingsById}(n_1,n_2)$
\EndIf
\EndProcedure
\\
\EndProcedure

\end{algorithmic}
\caption{Procedure to find relationships}
\label{AlgoGeneral}
\end{figure}

Our approach employs the procedure described in Figure~\ref{AlgoGeneral} to find the relationships.
The procedure \textsc{FindRelationships} has two parameters, $t_1$ and $t_2$, which are the root nodes of the CST's of both revisions.
Initially, we define $R \gets \emptyset$ as the set of relationships found so far.
Additionally, we also define $M \gets \emptyset$ as the set of pairs of matching nodes found so far.
Then, we execute four subroutines:
\begin{enumerate}

\item In \textsc{findMatchingsById}, we recursively look for matching nodes that have the same identifier and parent, i.e., we assume that code elements with the same identifier and parent are the same.
Such assumption enable us to match many code elements at this step, reducing the number of possibilities that need to be checked in the next steps.
The procedure consists of a loop that pairs the children of the nodes received as arguments and calls the procedure \textsc{addMatch} whenever a matching is found.
On its turn, \textsc{addMatch} adds a pair of matching nodes to $M$ and calls \textsc{findMatchingsById} again to look for matchings on their children, completing the recursion.
The matching pairs found in this step will be resolved to \textit{Same} and \textit{Convert type} relationships later.

\item In \textsc{findMatchingsBySim}, we look for matching nodes based on code similarity.
This is the case of \textit{Change signature}, \textit{Move}, \textit{Rename}, and \textit{Move and Rename} relationships.
The procedure starts by computing the set $M'$, which contains unmatched pairs of nodes from $t_1$ and $t_2$.
We use the notation $N^-$ to denote the set of unmatched nodes from $t_1$ (presumably deleted) and $N^+$ to denote the set of unmatched nodes from $t_2$ (presumably added).
Then, it repeatedly takes a pair $(n_1, n_2)$ from $M'$, sorting by the $\rdrank$ function, and checks if it meets the conditions (specified in the second column of Table~\ref{TabRelationshipTypes}) for any matching relationship by calling $\mathit{findMatchRel}(n_1, n_2)$.
That function will either return a singleton containing a matching relationship or an empty set if none of the conditions are met.
Last, the \textsc{addMatch} subroutine is called in the case a matching.
It's worth noting that the conditions to find those relationships and the $\rdrank$ function relies on a code similarity metric, which is described in details in Section~\ref{SecCodeSim}.

\item In \textsc{resolveMatchings}, we add the relationships corresponding to the matching pairs found at steps~1 and 2 to $R$.
The procedure iterates over the elements of $M$ and calls $\mathit{findMatchRel}$ to find which relationship type holds between $n_1$ and $n_2$.
By the end of this step, $R$ will contain all matching relationships found.
The rationale for postponing the resolution of the relationship type is discussed in Section~\ref{SecDependentConflictingRel}.

\item In \textsc{findNonMatchingRel}, we look for non-matching relationships.
First, we iterate over the pairs of matched/unmatched nodes, i.e., $M_1 \times N^+$, to look for \textit{ExtractSupertype}, \textit{Extract} and \textit{Extract and Move} relationships.
Similarly, we also iterate over the pairs of unmatched/matched nodes ($N^- \times M_2$) to look for \textit{Inline} relationships.
The functions $findExtractSupertype$, $findExtract$, $findExtractMove$, and $findInline$ checks the  preconditions for the corresponding relationship types, according to Table~\ref{TabRelationshipTypes}.
By the end of this last step, $R$ will contain all relationships found.
\end{enumerate}


Figure~\ref{FigRelationships1} shows the relationships we would find after running RefDiff in the example system from Figure~\ref{FigDiff1}.
Each relationship is represented by and edge connecting nodes from the left and right CST's.
There are three relationships of the type \textit{Same}, involving the code elements whose identifiers didn't change: the class \codeinl{Main} and the methods \codeinl{main} and \codeinl{sum}.
Two of the relationships are of the type \textit{Rename}, indicating that the class \codeinl{Calculator} was renamed to \codeinl{FpCalculator}, and the method \codeinl{min} was renamed to \codeinl{minimum}.
Moreover, there is an \textit{Extract} relationship indicating that the method \codeinl{print} was extract from \codeinl{main}.
Finally, we can also note that two nodes, $n_8$ and $n_{12}$, are not involved in matching relationships. Thus, we classify them as added code elements. As every node on the left side is matched, there are no deleted code elements.


\begin{figure}[htb]
\centering
\includegraphics[width=1.0\linewidth]{img/relationshipDiff1.pdf}
\caption{Relationships found in the example system from Figure~\ref{FigDiff1}}
\label{FigRelationships1}
\end{figure}


\subsubsection{Dependent and conflicting relationships}
\label{SecDependentConflictingRel}

In some cases, correctly finding a relationship depends on finding a prior relationship.
For example, consider the relationship $(n_5, n_{11}, Rename)$ in Figure~\ref{FigRelationships1} (method \codeinl{min} renamed to \codeinl{minimum}).
The conditions for this relationship includes the clause $\rdparent(n_5)' = \rdparent(n_{11})$, which means that the matching node of the parent of $n_5$ should be equal to the parent of $n_{11}$.
This clause only yields true after the matching pair $(n_2, n_9)$ is added to $M$, i.e., after we find out that \codeinl{Calculator} was renamed to \codeinl{FpCalculator}.
In fact, if we call $\mathit{findMatchRel}(n_5, n_{11})$ before $M$ contains $(n_2, n_9)$, we would incorrectly classify it as a \textit{Move and Rename} relationship.
To address this issue, we only resolve the actual relationship type in step~3, after all matching pairs are found (note that in steps~1 and 2 we record the matching pairs in $M$, purposely ignoring the type of relationship).
Moreover, non-matching relationships, which also depends on the matchings found, are analyzed in the last step. 

Another issue which we may face when looking for relationships are conflicts, i.e., the conditions for different matching relationships involving the same code element may yield true.
For instance, in the example from Figure~\ref{FigRelationships1}, the conditions for \textit{Rename} yield true for the pair of methods \codeinl{min} and \codeinl{minimum} because their source code are similar and their parents match.
However, this is also the case for the pair of methods \codeinl{min} and \codeinl{maximum}.
We cannot match the same node twice, thus, we must decide upon which relationship we will accept and discard the other one.
This issue is addressed in the \textsc{findMatchingsBySim} procedure by using the $\rdrank$ function to sort the potential matching pairs, enforcing that we take first the most likely matches.
After a matching pair $(n_1, n_2)$ is added to $M$, no more matchings involving $n_1$ or $n_2$ will be accepted, because \textsc{addMatch} procedure checks that $n_1 \in N^- \land n_2 \in N^+$.

\subsection{Code Similarity}
\label{SecCodeSim}

\begin{figure*}[htb]
\renewcommand{\arraystretch}{1.3}
\centering
\footnotesize
\begin{tabular}{@{}llll@{}}
\begin{tabular}{p{6.5cm}}
\multicolumn{1}{c}{\textbf{Source code of a class}} \\
\begin{lstlisting}
public class Calculator {

  public int sum(int x, int y) {
    return x + y;
  }

  public int min(int x, int y) {
    if (x < y) return x;
    else return y;
  }

  public double power(int b, int e) {
    return Math.pow(b, e);
  }
}
\end{lstlisting}\\

\end{tabular}
& {\Large $\Rightarrow$} &
\begin{tabular}{|l|r|r|r|}
\multicolumn{4}{c}{\textbf{Multiset of tokens for each method}} \\
\hline
Token $t$ & $m_{\mathtt{sum}}(t)$ & $m_{\mathtt{min}}(t)$ & $m_{\mathtt{power}}(t)$\\
\hline
\codeinl{return} & 1 & 2 & 1 \\
\codeinl{x}      & 1 & 2 & 0 \\
\codeinl{+}      & 1 & 0 & 0 \\
\codeinl{y}      & 1 & 2 & 0 \\
\codeinl{;}      & 1 & 2 & 1 \\
\codeinl{if}     & 0 & 1 & 0 \\
\codeinl{(}      & 0 & 1 & 1 \\
\codeinl{<}      & 0 & 1 & 0 \\
\codeinl{)}      & 0 & 1 & 1 \\
\codeinl{else}   & 0 & 1 & 0 \\
\codeinl{Math}   & 0 & 0 & 1 \\
\codeinl{.}      & 0 & 0 & 1 \\
\codeinl{pow}    & 0 & 0 & 1 \\
\codeinl{b}      & 0 & 0 & 1 \\
\codeinl{,}      & 0 & 0 & 1 \\
\codeinl{e}      & 0 & 0 & 1 \\
\hline
\end{tabular} &
\begin{tabular}{|r|}
\multicolumn{1}{c}{} \\
\hline
$n_t$\\
\hline
3 \\
2 \\
1 \\
2 \\
3 \\
1 \\
2 \\
1 \\
2 \\
1 \\
1 \\
1 \\
1 \\
1 \\
1 \\
1 \\
\hline
\end{tabular}
\end{tabular}
\caption{Transformation of the body of each method into a multiset of tokens}
\label{FigSourceCodeTransformation}
\end{figure*}

A key element of our approach to find relationships, as mentioned previously, is computing the similarity.
The first step to compute the similarity between code elements is to represent their source code as a multiset (or bag) of tokens.
A multiset is a generalization of the concept of a set, but it allows multiple instances of the same element.
The multiplicity of an element is the number of occurrences of that element within the multiset. Formally, a multiset can be defined in terms of a multiplicity function $m: U \to \mathbb{N}$, where $U$ is the set of all possible elements. In other words, $m(t)$ is the multiplicity of the element $t$ in the multiset. Note that the multiplicity of an element that is not in the multiset is zero.

For example, Figure~\ref{FigSourceCodeTransformation} depicts the transformation of the source code of three methods (\codeinl{sum}, \codeinl{min}, and \codeinl{power}), of the class \codeinl{Calculator}, into multisets of tokens. In this figure, the multiplicity function $m$ for each method is represented in a tabular form. For example, the multiplicity of the token \codeinl{y} in method \codeinl{min} is two (i.e., $m_{\mathtt{min}}(\mathtt{y}) = 2$), whilst the multiplicity of the token \codeinl{if} in method \codeinl{power} is zero (i.e., $m_{\mathtt{power}}(\mathtt{if}) = 0$).


Later, to compute the similarity between two code elements $e_1$ and $e_2$, we use a generalization of the Jaccard coefficient, known as weighted Jaccard coefficient~\cite{chierichetti2010finding}.
Let $U$ be the set of all possible tokens and $w(e, t)$ be a weight function of a token $t$ for the entity $e$.
We define the similarity between $e_1$ and $e_2$ by the following formula:


\begin{align}
\rdsim(e_1, e_2) = \frac{\sum_{t \in U} \min(w(e_1, t), w(e_2, t))}
                        {\sum_{t \in U} \max(w(e_1, t), w(e_2, t))}
\end{align}

%\subsubsubsection{Weight of a token for a code entity}

Our similarity function is based on a weighting function $w(e, t)$ that expresses the importance a token $t$ for a code entity $e$.
In fact, some tokens are more important than others to discriminate a code element.
For example, in Figure~\ref{FigSourceCodeTransformation}, all three methods contain the token \codeinl{return}. In contrast, only one method (\codeinl{power}) contains the token \codeinl{Math}. Therefore, the later is a better indicator of similarity between methods than the former.

In order to take this into account, we employ a variation of the TF-IDF weighting scheme~\cite{salton1986introduction}, which is a well-known technique from information retrieval.
TF-IDF, which is the short form of \emph{Term Frequency–Inverse Document Frequency}, reflects how important a term is to a document within a collection of documents.
In the context of code elements, we consider a token as a term, and the body of a method (or class) as a document.

Let $E$ be the set of all code elements and $n_t$ be the number of elements in $E$ that contains the token $t$,
we define the weight of $t$ for a code element $e$ as the function $w(e, t)$, which is defined by the following formula:
\begin{align}
w(e, t) = m_e(t) \times \mathit{idf}(t)
\end{align}

\noindent where $m_e(t)$ is the multiplicity of $t$ in $e$, and $\mathit{idf}(t)$ is the Inverse Document Frequency, which is defined as:
\begin{align}
\mathit{idf}(t) = \log (1 + \frac{|E|}{n_t})
\end{align}

Note that the value of $\mathit{idf}(t)$ decreases as $n_t$ increases, because the more frequent a token is among the collection of code elements, the less important it is to distinguish code elements.
For example, in Figure~\ref{FigSourceCodeTransformation}, the token \codeinl{y} occurs in two methods (\codeinl{sum} and \codeinl{min}). Thus, its $\mathit{idf}$ is:

\[
\mathit{idf}(\mathtt{y}) = 
\log (1 + \frac{|E|}{n_t}) = 
\log (1 + \frac{3}{2}) = 0.398
\]

On the other hand, the token \codeinl{else} occurs in one method ($\mathtt{min}$), and its $\mathit{idf}$ is:

\[
\mathit{idf}(\mathtt{else}) = 
\log (1 + \frac{|E|}{n_t}) = 
\log (1 + \frac{3}{1}) = 0.602
\]


\subsubsection{Extract and Inline Similarity}
\label{SecSimX}

While the similarity function presented previously is suitable to compute whether two code elements are are similar, in some situations we need to assess whether the source code of an element is contained within another one. This is the case of \emph{Extract}, and \emph{Inline} relationships. For example, if a method $e_2$ is extracted from $e_1$, the source code of $e_2$ should be contained within $e_1$ prior to the refactoring, although $e_1$ and $e_2$ may be significantly different from each other. Analogously, if a method $e_1$ is inlined into $e_2$, the source code of $e_1$ should be contained within $e_2$.
Thus, we defined a specialized version of the similarity function $\rdsimx$ defined by the following formula:
\begin{align}
\rdsimx(e_1, e_2) = \frac{\sum_{t \in U} \min(w(e_1, t), w(e_2, t))}
                        {\sum_{t \in U} w(e_1, t)}
\end{align}

The rationale behind this formula is that the similarity is at maximum when the multiset of tokens of $e_1$ is a subset of the multiset of tokens of $e_2$, i.e., all tokens from $e_1$ can be found in $e_2$. Note that, given this definition, $\rdsimx(e_1, e_2) \neq \rdsimx(e_2, e_1)$.


