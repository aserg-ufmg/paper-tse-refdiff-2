\documentclass[10pt,journal,compsoc]{IEEEtran}

\usepackage[utf8]{inputenc}


\usepackage{booktabs}
\usepackage[usenames]{color}
\usepackage{blindtext}


\definecolor{darkgray}{RGB}{90,90,90}
\definecolor{lightgray}{RGB}{210,210,210}
\newcommand\xbar[1]{#1 {\color{darkgray} \rule{\dimexpr #1pt * 16}{5.5pt}}{\color{lightgray} \rule{\dimexpr 16pt - (#1pt * 16)}{5.5pt}}}

\newcommand\xbard[3]{#1/#2 {\color{darkgray} \rule{\dimexpr #3pt * 16}{5.5pt}}{\color{lightgray} \rule{\dimexpr 16pt - (#1pt * 16)}{5.5pt}}}

\definecolor{codegray}{gray}{0.9}
\newcommand{\code}[1]{\colorbox{codegray}{\texttt{#1}}}
\newcommand{\codeinl}[1]{%
  \begingroup
  \setlength{\fboxsep}{2pt}%
  \mbox{\vphantom{#1}\smash{\colorbox{codegray}{\texttt{#1}}}}%
  \endgroup
}

\newcommand{\todo}[1]{\textcolor{red}{TODO: #1}}

% Some very useful LaTeX packages include:
% (uncomment the ones you want to load)


% *** MISC UTILITY PACKAGES ***
%
%\usepackage{ifpdf}
% Heiko Oberdiek's ifpdf.sty is very useful if you need conditional
% compilation based on whether the output is pdf or dvi.
% usage:
% \ifpdf
%   % pdf code
% \else
%   % dvi code
% \fi
% The latest version of ifpdf.sty can be obtained from:
% http://www.ctan.org/pkg/ifpdf
% Also, note that IEEEtran.cls V1.7 and later provides a builtin
% \ifCLASSINFOpdf conditional that works the same way.
% When switching from latex to pdflatex and vice-versa, the compiler may
% have to be run twice to clear warning/error messages.






% *** CITATION PACKAGES ***
%
\ifCLASSOPTIONcompsoc
  % IEEE Computer Society needs nocompress option
  % requires cite.sty v4.0 or later (November 2003)
  \usepackage[nocompress]{cite}
\else
  % normal IEEE
  \usepackage{cite}
\fi
% cite.sty was written by Donald Arseneau
% V1.6 and later of IEEEtran pre-defines the format of the cite.sty package
% \cite{} output to follow that of the IEEE. Loading the cite package will
% result in citation numbers being automatically sorted and properly
% "compressed/ranged". e.g., [1], [9], [2], [7], [5], [6] without using
% cite.sty will become [1], [2], [5]--[7], [9] using cite.sty. cite.sty's
% \cite will automatically add leading space, if needed. Use cite.sty's
% noadjust option (cite.sty V3.8 and later) if you want to turn this off
% such as if a citation ever needs to be enclosed in parenthesis.
% cite.sty is already installed on most LaTeX systems. Be sure and use
% version 5.0 (2009-03-20) and later if using hyperref.sty.
% The latest version can be obtained at:
% http://www.ctan.org/pkg/cite
% The documentation is contained in the cite.sty file itself.
%
% Note that some packages require special options to format as the Computer
% Society requires. In particular, Computer Society  papers do not use
% compressed citation ranges as is done in typical IEEE papers
% (e.g., [1]-[4]). Instead, they list every citation separately in order
% (e.g., [1], [2], [3], [4]). To get the latter we need to load the cite
% package with the nocompress option which is supported by cite.sty v4.0
% and later. Note also the use of a CLASSOPTION conditional provided by
% IEEEtran.cls V1.7 and later.





% *** GRAPHICS RELATED PACKAGES ***
%
\ifCLASSINFOpdf
  \usepackage[pdftex]{graphicx}
  % declare the path(s) where your graphic files are
  % \graphicspath{{../pdf/}{../jpeg/}}
  % and their extensions so you won't have to specify these with
  % every instance of \includegraphics
  % \DeclareGraphicsExtensions{.pdf,.jpeg,.png}
\else
  % or other class option (dvipsone, dvipdf, if not using dvips). graphicx
  % will default to the driver specified in the system graphics.cfg if no
  % driver is specified.
  \usepackage[dvips]{graphicx}
  % declare the path(s) where your graphic files are
  % \graphicspath{{../eps/}}
  % and their extensions so you won't have to specify these with
  % every instance of \includegraphics
  % \DeclareGraphicsExtensions{.eps}
\fi
% graphicx was written by David Carlisle and Sebastian Rahtz. It is
% required if you want graphics, photos, etc. graphicx.sty is already
% installed on most LaTeX systems. The latest version and documentation
% can be obtained at: 
% http://www.ctan.org/pkg/graphicx
% Another good source of documentation is "Using Imported Graphics in
% LaTeX2e" by Keith Reckdahl which can be found at:
% http://www.ctan.org/pkg/epslatex
%
% latex, and pdflatex in dvi mode, support graphics in encapsulated
% postscript (.eps) format. pdflatex in pdf mode supports graphics
% in .pdf, .jpeg, .png and .mps (metapost) formats. Users should ensure
% that all non-photo figures use a vector format (.eps, .pdf, .mps) and
% not a bitmapped formats (.jpeg, .png). The IEEE frowns on bitmapped formats
% which can result in "jaggedy"/blurry rendering of lines and letters as
% well as large increases in file sizes.
%
% You can find documentation about the pdfTeX application at:
% http://www.tug.org/applications/pdftex






% *** MATH PACKAGES ***
%
\usepackage{amsmath}
% A popular package from the American Mathematical Society that provides
% many useful and powerful commands for dealing with mathematics.
%
% Note that the amsmath package sets \interdisplaylinepenalty to 10000
% thus preventing page breaks from occurring within multiline equations. Use:
%\interdisplaylinepenalty=2500
% after loading amsmath to restore such page breaks as IEEEtran.cls normally
% does. amsmath.sty is already installed on most LaTeX systems. The latest
% version and documentation can be obtained at:
% http://www.ctan.org/pkg/amsmath

\usepackage{amssymb}

\DeclareMathOperator{\rdname}{name}
\DeclareMathOperator{\rdparent}{\pi}
\DeclareMathOperator{\rdsig}{id}
\DeclareMathOperator{\rdns}{ns}
\DeclareMathOperator{\rdsub}{subtype}
\DeclareMathOperator{\rdtype}{nType}
\DeclareMathOperator{\rdsim}{sim}
\DeclareMathOperator{\rdnsim}{nameSim}
\DeclareMathOperator{\rdsimx}{sim_{x}}
\DeclareMathOperator{\rdsimi}{sim_{i}}
\DeclareMathOperator{\rdsimc}{sim_{\subseteq}}
\DeclareMathOperator{\rduses}{uses}
\DeclareMathOperator{\rdchildren}{childr}
\DeclareMathOperator{\rdsortbysim}{sortBySim}

\DeclareMathOperator{\rdSame}{Same}
\DeclareMathOperator{\rdConvertType}{ConvertType}
\DeclareMathOperator{\rdExtract}{Extract}
\DeclareMathOperator{\rdExtractSuper}{ExtractSuper}
\DeclareMathOperator{\rdInline}{Inline}
\DeclareMathOperator{\rdrank}{\mathit{rank}}

\DeclareMathOperator*{\argmin}{arg\,min}


\usepackage{listings}
\usepackage{lipsum}
\usepackage{courier}
\definecolor{javablue}{rgb}{0.25,0,1} % for strings
\definecolor{javagreen}{rgb}{0.25,0.5,0.35} % comments
\definecolor{javapurple}{rgb}{0.5,0,0.35} % keywords
\definecolor{javadocblue}{rgb}{0.25,0.35,0.75} % javadoc
\lstset{
language=Java,
basicstyle=\footnotesize\ttfamily,
breaklines=true,
keywordstyle=\color{javapurple}\bfseries,
stringstyle=\color{javablue},
commentstyle=\color{javagreen},
morecomment=[s][\color{javadocblue}]{/**}{*/},
tabsize=2,
frame=single}


\usepackage{hyperref}

% *** SPECIALIZED LIST PACKAGES ***
%
%\usepackage{algorithmic}
% algorithmic.sty was written by Peter Williams and Rogerio Brito.
% This package provides an algorithmic environment fo describing algorithms.
% You can use the algorithmic environment in-text or within a figure
% environment to provide for a floating algorithm. Do NOT use the algorithm
% floating environment provided by algorithm.sty (by the same authors) or
% algorithm2e.sty (by Christophe Fiorio) as the IEEE does not use dedicated
% algorithm float types and packages that provide these will not provide
% correct IEEE style captions. The latest version and documentation of
% algorithmic.sty can be obtained at:
% http://www.ctan.org/pkg/algorithms
% Also of interest may be the (relatively newer and more customizable)
% algorithmicx.sty package by Szasz Janos:
% http://www.ctan.org/pkg/algorithmicx
\usepackage{algpseudocode}

\algnewcommand\algorithmicforeach{\textbf{for each}}
\algdef{S}[FOR]{ForEach}[1]{\algorithmicforeach\ #1\ \algorithmicdo}


% *** ALIGNMENT PACKAGES ***
%
\usepackage{array}
% Frank Mittelbach's and David Carlisle's array.sty patches and improves
% the standard LaTeX2e array and tabular environments to provide better
% appearance and additional user controls. As the default LaTeX2e table
% generation code is lacking to the point of almost being broken with
% respect to the quality of the end results, all users are strongly
% advised to use an enhanced (at the very least that provided by array.sty)
% set of table tools. array.sty is already installed on most systems. The
% latest version and documentation can be obtained at:
% http://www.ctan.org/pkg/array


% IEEEtran contains the IEEEeqnarray family of commands that can be used to
% generate multiline equations as well as matrices, tables, etc., of high
% quality.




% *** SUBFIGURE PACKAGES ***
%\ifCLASSOPTIONcompsoc
%  \usepackage[caption=false,font=footnotesize,labelfont=sf,textfont=sf]{subfig}
%\else
%  \usepackage[caption=false,font=footnotesize]{subfig}
%\fi
% subfig.sty, written by Steven Douglas Cochran, is the modern replacement
% for subfigure.sty, the latter of which is no longer maintained and is
% incompatible with some LaTeX packages including fixltx2e. However,
% subfig.sty requires and automatically loads Axel Sommerfeldt's caption.sty
% which will override IEEEtran.cls' handling of captions and this will result
% in non-IEEE style figure/table captions. To prevent this problem, be sure
% and invoke subfig.sty's "caption=false" package option (available since
% subfig.sty version 1.3, 2005/06/28) as this is will preserve IEEEtran.cls
% handling of captions.
% Note that the Computer Society format requires a sans serif font rather
% than the serif font used in traditional IEEE formatting and thus the need
% to invoke different subfig.sty package options depending on whether
% compsoc mode has been enabled.
%
% The latest version and documentation of subfig.sty can be obtained at:
% http://www.ctan.org/pkg/subfig




% *** FLOAT PACKAGES ***
%
%\usepackage{fixltx2e}
% fixltx2e, the successor to the earlier fix2col.sty, was written by
% Frank Mittelbach and David Carlisle. This package corrects a few problems
% in the LaTeX2e kernel, the most notable of which is that in current
% LaTeX2e releases, the ordering of single and double column floats is not
% guaranteed to be preserved. Thus, an unpatched LaTeX2e can allow a
% single column figure to be placed prior to an earlier double column
% figure.
% Be aware that LaTeX2e kernels dated 2015 and later have fixltx2e.sty's
% corrections already built into the system in which case a warning will
% be issued if an attempt is made to load fixltx2e.sty as it is no longer
% needed.
% The latest version and documentation can be found at:
% http://www.ctan.org/pkg/fixltx2e


%\usepackage{stfloats}
% stfloats.sty was written by Sigitas Tolusis. This package gives LaTeX2e
% the ability to do double column floats at the bottom of the page as well
% as the top. (e.g., "\begin{figure*}[!b]" is not normally possible in
% LaTeX2e). It also provides a command:
%\fnbelowfloat
% to enable the placement of footnotes below bottom floats (the standard
% LaTeX2e kernel puts them above bottom floats). This is an invasive package
% which rewrites many portions of the LaTeX2e float routines. It may not work
% with other packages that modify the LaTeX2e float routines. The latest
% version and documentation can be obtained at:
% http://www.ctan.org/pkg/stfloats
% Do not use the stfloats baselinefloat ability as the IEEE does not allow
% \baselineskip to stretch. Authors submitting work to the IEEE should note
% that the IEEE rarely uses double column equations and that authors should try
% to avoid such use. Do not be tempted to use the cuted.sty or midfloat.sty
% packages (also by Sigitas Tolusis) as the IEEE does not format its papers in
% such ways.
% Do not attempt to use stfloats with fixltx2e as they are incompatible.
% Instead, use Morten Hogholm'a dblfloatfix which combines the features
% of both fixltx2e and stfloats:
%
% \usepackage{dblfloatfix}
% The latest version can be found at:
% http://www.ctan.org/pkg/dblfloatfix




%\ifCLASSOPTIONcaptionsoff
%  \usepackage[nomarkers]{endfloat}
% \let\MYoriglatexcaption\caption
% \renewcommand{\caption}[2][\relax]{\MYoriglatexcaption[#2]{#2}}
%\fi
% endfloat.sty was written by James Darrell McCauley, Jeff Goldberg and 
% Axel Sommerfeldt. This package may be useful when used in conjunction with 
% IEEEtran.cls'  captionsoff option. Some IEEE journals/societies require that
% submissions have lists of figures/tables at the end of the paper and that
% figures/tables without any captions are placed on a page by themselves at
% the end of the document. If needed, the draftcls IEEEtran class option or
% \CLASSINPUTbaselinestretch interface can be used to increase the line
% spacing as well. Be sure and use the nomarkers option of endfloat to
% prevent endfloat from "marking" where the figures would have been placed
% in the text. The two hack lines of code above are a slight modification of
% that suggested by in the endfloat docs (section 8.4.1) to ensure that
% the full captions always appear in the list of figures/tables - even if
% the user used the short optional argument of \caption[]{}.
% IEEE papers do not typically make use of \caption[]'s optional argument,
% so this should not be an issue. A similar trick can be used to disable
% captions of packages such as subfig.sty that lack options to turn off
% the subcaptions:
% For subfig.sty:
% \let\MYorigsubfloat\subfloat
% \renewcommand{\subfloat}[2][\relax]{\MYorigsubfloat[]{#2}}
% However, the above trick will not work if both optional arguments of
% the \subfloat command are used. Furthermore, there needs to be a
% description of each subfigure *somewhere* and endfloat does not add
% subfigure captions to its list of figures. Thus, the best approach is to
% avoid the use of subfigure captions (many IEEE journals avoid them anyway)
% and instead reference/explain all the subfigures within the main caption.
% The latest version of endfloat.sty and its documentation can obtained at:
% http://www.ctan.org/pkg/endfloat
%
% The IEEEtran \ifCLASSOPTIONcaptionsoff conditional can also be used
% later in the document, say, to conditionally put the References on a 
% page by themselves.




% *** PDF, URL AND HYPERLINK PACKAGES ***
%
%\usepackage{url}
% url.sty was written by Donald Arseneau. It provides better support for
% handling and breaking URLs. url.sty is already installed on most LaTeX
% systems. The latest version and documentation can be obtained at:
% http://www.ctan.org/pkg/url
% Basically, \url{my_url_here}.





% *** Do not adjust lengths that control margins, column widths, etc. ***
% *** Do not use packages that alter fonts (such as pslatex).         ***
% There should be no need to do such things with IEEEtran.cls V1.6 and later.
% (Unless specifically asked to do so by the journal or conference you plan
% to submit to, of course. )


% correct bad hyphenation here
\hyphenation{op-tical net-works semi-conduc-tor}
\hyphenation{RefDiff}


\begin{document}
%
% paper title
% Titles are generally capitalized except for words such as a, an, and, as,
% at, but, by, for, in, nor, of, on, or, the, to and up, which are usually
% not capitalized unless they are the first or last word of the title.
% Linebreaks \\ can be used within to get better formatting as desired.
% Do not put math or special symbols in the title.
\title{RefDiff 2.0: A Multi-language Refactoring Detection Tool}
%
%
% author names and IEEE memberships
% note positions of commas and nonbreaking spaces ( ~ ) LaTeX will not break
% a structure at a ~ so this keeps an author's name from being broken across
% two lines.
% use \thanks{} to gain access to the first footnote area
% a separate \thanks must be used for each paragraph as LaTeX2e's \thanks
% was not built to handle multiple paragraphs
%
%
%\IEEEcompsocitemizethanks is a special \thanks that produces the bulleted
% lists the Computer Society journals use for "first footnote" author
% affiliations. Use \IEEEcompsocthanksitem which works much like \item
% for each affiliation group. When not in compsoc mode,
% \IEEEcompsocitemizethanks becomes like \thanks and
% \IEEEcompsocthanksitem becomes a line break with idention. This
% facilitates dual compilation, although admittedly the differences in the
% desired content of \author between the different types of papers makes a
% one-size-fits-all approach a daunting prospect. For instance, compsoc 
% journal papers have the author affiliations above the "Manuscript
% received ..."  text while in non-compsoc journals this is reversed. Sigh.

\author{Danilo~Silva, %~\IEEEmembership{Member,~IEEE,}
        João~Paulo~da~Silva, %~\IEEEmembership{Fellow,~OSA,}
        Gustavo~Santos, %~\IEEEmembership{Fellow,~OSA,}
        Ricardo~Terra, %~\IEEEmembership{Fellow,~OSA,}
        and~Marco~Tulio~Valente,~\IEEEmembership{Member,~IEEE}% <-this % stops a space
\IEEEcompsocitemizethanks{
\IEEEcompsocthanksitem D. Silva and M.T. Valente are with the 
Department of Computer Science, Federal University of Minas Gerais, Av. Antônio Carlos, Belo Horizonte, MG 31270-010, Brazil.\protect\\
E-mail: \{danilofs, mtov\}@dcc.ufmg.br.
\IEEEcompsocthanksitem J.P. da Silva is a developer and team lead at Quimbik, Inc.\protect\\
E-mail: joao@jpribeiro.com.br.
\IEEEcompsocthanksitem G. Santos is with Federal University of Technology (UTFPR), Dois Vizinhos, PR 85660-000, Brazil.\protect\\
E-mail: gustavosantos@utfpr.edu.br.
\IEEEcompsocthanksitem R. Terra is with the Department of Computer Science, Federal University of Lavras (UFLA), Lavras, MG 37200-000, Brazil.\protect\\
E-mail: terra@dcc.ufla.br}% <-this % stops an unwanted space
%\thanks{Manuscript received April 19, 2005; revised August 26, 2015.}
}

% note the % following the last \IEEEmembership and also \thanks - 
% these prevent an unwanted space from occurring between the last author name
% and the end of the author line. i.e., if you had this:
% 
% \author{....lastname \thanks{...} \thanks{...} }
%                     ^------------^------------^----Do not want these spaces!
%
% a space would be appended to the last name and could cause every name on that
% line to be shifted left slightly. This is one of those "LaTeX things". For
% instance, "\textbf{A} \textbf{B}" will typeset as "A B" not "AB". To get
% "AB" then you have to do: "\textbf{A}\textbf{B}"
% \thanks is no different in this regard, so shield the last } of each \thanks
% that ends a line with a % and do not let a space in before the next \thanks.
% Spaces after \IEEEmembership other than the last one are OK (and needed) as
% you are supposed to have spaces between the names. For what it is worth,
% this is a minor point as most people would not even notice if the said evil
% space somehow managed to creep in.



% The paper headers
\markboth{IEEE Transactions on Software Engineering,~Vol.~XX, No.~X, August~XXXX}%
{Silva \MakeLowercase{\textit{et al.}}: RefDiff 2.0: A Multi-language Refactoring Detection Tool}
% The only time the second header will appear is for the odd numbered pages
% after the title page when using the twoside option.
% 
% *** Note that you probably will NOT want to include the author's ***
% *** name in the headers of peer review papers.                   ***
% You can use \ifCLASSOPTIONpeerreview for conditional compilation here if
% you desire.



% The publisher's ID mark at the bottom of the page is less important with
% Computer Society journal papers as those publications place the marks
% outside of the main text columns and, therefore, unlike regular IEEE
% journals, the available text space is not reduced by their presence.
% If you want to put a publisher's ID mark on the page you can do it like
% this:
%\IEEEpubid{0000--0000/00\$00.00~\copyright~2015 IEEE}
% or like this to get the Computer Society new two part style.
%\IEEEpubid{\makebox[\columnwidth]{\hfill 0000--0000/00/\$00.00~\copyright~2015 IEEE}%
%\hspace{\columnsep}\makebox[\columnwidth]{Published by the IEEE Computer Society\hfill}}
% Remember, if you use this you must call \IEEEpubidadjcol in the second
% column for its text to clear the IEEEpubid mark (Computer Society jorunal
% papers don't need this extra clearance.)



% use for special paper notices
%\IEEEspecialpapernotice{(Invited Paper)}



% for Computer Society papers, we must declare the abstract and index terms
% PRIOR to the title within the \IEEEtitleabstractindextext IEEEtran
% command as these need to go into the title area created by \maketitle.
% As a general rule, do not put math, special symbols or citations
% in the abstract or keywords.
\IEEEtitleabstractindextext{%
\begin{abstract}
Identifying refactoring operations in source code changes is valuable to understand software evolution.
Therefore, several tools have been proposed to automatically detect refactorings applied in a system by comparing source code between revisions.
The availability of such infrastructure has enabled researchers to study refactoring practice in large scale, leading to important advances on refactoring knowledge.
However, although a plethora of programming languages are used in practice, the vast majority of existing studies are restricted to the Java language due to limitations of the underlying tools.
This fact poses an important threat to external validity.
Thus, to overcome such limitation, in this paper we propose RefDiff~2.0, a multi-language refactoring detection tool.
Our approach leverages techniques proposed in our previous work and introduces a novel refactoring detection algorithm that relies on the Code Structure Tree (CST), a simple yet powerful representation of the source code that abstracts away the specificities of particular programming languages.
Despite its language-agnostic design, our evaluation shows that RefDiff's precision (96\%) and recall (80\%) are on par with state-of-the-art refactoring detection approaches specialized in the Java language.
Our modular architecture also enables one to seamlessly extend RefDiff to support other languages via a plugin system.
As a proof of this, we implemented plugins to support two other popular programming languages: JavaScript and C.
Our evaluation in these languages reveals that precision and recall ranges from 88\% to 91\%.
With these results, we envision RefDiff as a viable alternative for breaking the single-language barrier in refactoring research and in practical applications of refactoring detection.


\end{abstract}

% Note that keywords are not normally used for peerreview papers.
%\begin{IEEEkeywords}
%Computer Society, IEEE, IEEEtran, journal, \LaTeX, paper, template.
%\end{IEEEkeywords}
}


% make the title area
\maketitle


% To allow for easy dual compilation without having to reenter the
% abstract/keywords data, the \IEEEtitleabstractindextext text will
% not be used in maketitle, but will appear (i.e., to be "transported")
% here as \IEEEdisplaynontitleabstractindextext when the compsoc 
% or transmag modes are not selected <OR> if conference mode is selected 
% - because all conference papers position the abstract like regular
% papers do.
\IEEEdisplaynontitleabstractindextext
% \IEEEdisplaynontitleabstractindextext has no effect when using
% compsoc or transmag under a non-conference mode.



% For peer review papers, you can put extra information on the cover
% page as needed:
% \ifCLASSOPTIONpeerreview
% \begin{center} \bfseries EDICS Category: 3-BBND \end{center}
% \fi
%
% For peerreview papers, this IEEEtran command inserts a page break and
% creates the second title. It will be ignored for other modes.
\IEEEpeerreviewmaketitle






% needed in second column of first page if using \IEEEpubid
%\IEEEpubidadjcol


% An example of a floating figure using the graphicx package.
% Note that \label must occur AFTER (or within) \caption.
% For figures, \caption should occur after the \includegraphics.
% Note that IEEEtran v1.7 and later has special internal code that
% is designed to preserve the operation of \label within \caption
% even when the captionsoff option is in effect. However, because
% of issues like this, it may be the safest practice to put all your
% \label just after \caption rather than within \caption{}.
%
% Reminder: the "draftcls" or "draftclsnofoot", not "draft", class
% option should be used if it is desired that the figures are to be
% displayed while in draft mode.
%
%\begin{figure}[!t]
%\centering
%\includegraphics[width=2.5in]{myfigure}
% where an .eps filename suffix will be assumed under latex, 
% and a .pdf suffix will be assumed for pdflatex; or what has been declared
% via \DeclareGraphicsExtensions.
%\caption{Simulation results for the network.}
%\label{fig_sim}
%\end{figure}

% Note that the IEEE typically puts floats only at the top, even when this
% results in a large percentage of a column being occupied by floats.
% However, the Computer Society has been known to put floats at the bottom.


% An example of a double column floating figure using two subfigures.
% (The subfig.sty package must be loaded for this to work.)
% The subfigure \label commands are set within each subfloat command,
% and the \label for the overall figure must come after \caption.
% \hfil is used as a separator to get equal spacing.
% Watch out that the combined width of all the subfigures on a 
% line do not exceed the text width or a line break will occur.
%
%\begin{figure*}[!t]
%\centering
%\subfloat[Case I]{\includegraphics[width=2.5in]{box}%
%\label{fig_first_case}}
%\hfil
%\subfloat[Case II]{\includegraphics[width=2.5in]{box}%
%\label{fig_second_case}}
%\caption{Simulation results for the network.}
%\label{fig_sim}
%\end{figure*}
%
% Note that often IEEE papers with subfigures do not employ subfigure
% captions (using the optional argument to \subfloat[]), but instead will
% reference/describe all of them (a), (b), etc., within the main caption.
% Be aware that for subfig.sty to generate the (a), (b), etc., subfigure
% labels, the optional argument to \subfloat must be present. If a
% subcaption is not desired, just leave its contents blank,
% e.g., \subfloat[].


% An example of a floating table. Note that, for IEEE style tables, the
% \caption command should come BEFORE the table and, given that table
% captions serve much like titles, are usually capitalized except for words
% such as a, an, and, as, at, but, by, for, in, nor, of, on, or, the, to
% and up, which are usually not capitalized unless they are the first or
% last word of the caption. Table text will default to \footnotesize as
% the IEEE normally uses this smaller font for tables.
% The \label must come after \caption as always.
%
%\begin{table}[!t]
%% increase table row spacing, adjust to taste
%\renewcommand{\arraystretch}{1.3}
% if using array.sty, it might be a good idea to tweak the value of
% \extrarowheight as needed to properly center the text within the cells
%\caption{An Example of a Table}
%\label{table_example}
%\centering
%% Some packages, such as MDW tools, offer better commands for making tables
%% than the plain LaTeX2e tabular which is used here.
%\begin{tabular}{|c||c|}
%\hline
%One & Two\\
%\hline
%Three & Four\\
%\hline
%\end{tabular}
%\end{table}


% Note that the IEEE does not put floats in the very first column
% - or typically anywhere on the first page for that matter. Also,
% in-text middle ("here") positioning is typically not used, but it
% is allowed and encouraged for Computer Society conferences (but
% not Computer Society journals). Most IEEE journals/conferences use
% top floats exclusively. 
% Note that, LaTeX2e, unlike IEEE journals/conferences, places
% footnotes above bottom floats. This can be corrected via the
% \fnbelowfloat command of the stfloats package.


\IEEEraisesectionheading{\section{Introduction}\label{sec:introduction}}
% Computer Society journal (but not conference!) papers do something unusual
% with the very first section heading (almost always called "Introduction").
% They place it ABOVE the main text! IEEEtran.cls does not automatically do
% this for you, but you can achieve this effect with the provided
% \IEEEraisesectionheading{} command. Note the need to keep any \label that
% is to refer to the section immediately after \section in the above as
% \IEEEraisesectionheading puts \section within a raised box.




% The very first letter is a 2 line initial drop letter followed
% by the rest of the first word in caps (small caps for compsoc).
% 
% form to use if the first word consists of a single letter:
% \IEEEPARstart{A}{demo} file is ....
% 
% form to use if you need the single drop letter followed by
% normal text (unknown if ever used by the IEEE):
% \IEEEPARstart{A}{}demo file is ....
% 
% Some journals put the first two words in caps:
% \IEEEPARstart{T}{his demo} file is ....
% 
% Here we have the typical use of a "T" for an initial drop letter
% and "HIS" in caps to complete the first word.
\IEEEPARstart{R}{efactoring} is a well-known technique to improve the design of a system and enable its evolution~\cite{Fowler:1999}.
In fact, existing studies~\cite{MurphyHill2012, tsantalis_empiricalstudy, Kim:2012:FSE, kim-tse-2014, fse2016-why-we-refactor} present strong evidences that refactoring is frequently applied by development teams, and it is an important aspect of their software maintenance workflow.


Therefore, knowing about the refactoring activity in a code change is a valuable information to help researchers to understand software evolution.
For example, past studies have used such information to shed light on important aspects of refactoring practice, such as: how developers refactor~\cite{MurphyHill2012}, the usage of refactoring tools~\cite{negara2013, MurphyHill2012}, the motivations driving refactoring~\cite{Kim:2012:FSE, kim-tse-2014, fse2016-why-we-refactor}, the risks of refactoring~\cite{Kim:2012:FSE, kim-tse-2014, Kim:2011, weissgerber2006refactorings, bavota2012does}, and the impact of refactoring on code quality metrics~\cite{Kim:2012:FSE, kim-tse-2014}.
Moreover, knowing which refactoring operations were applied in the version history of a system may help in several practical tasks.
For example, in a study by Kim~et~al.~\cite{Kim:2012:FSE}, many developers mentioned the difficulties they face when reviewing or integrating code changes after large refactoring operations, which moves or renames several code elements. Thus, developers feel discouraged to refactor their code. If a tool is able to identify such refactoring operations, it can possibly resolve merge conflicts automatically. 
Moreover, diff visualization tools can also benefit from such information, presenting refactored code elements side-by-side with their corresponding version before the change.
Another application for such information is adapting client code to a refactored version of an API it uses~\cite{henkel2005catchup, Xing:2008:JDevAn}. If we are able to detect the refactorings that were applied to an API, we can replay them on the client code automatically.
\todo{Falar no paragrafo acima de papers mais recentes, como o paper ICSE do Andre~\cite{icse2018}}

%Although there are approaches capable of detecting refactorings automatically, there are still some issues that hinder their application. Specifically, the precision and recall of such approaches still need improvements.
%In this paper, we try to fill this gap by proposing RefDiff, an automated approach that identifies refactorings performed in the version history of a system.
%RefDiff employs a combination of heuristics based on static analysis and code similarity to detect 13 well-known refactoring types.
%When compared to existing approaches, RefDiff leverages existing techniques and also introduces some novel ideas, such as the adaptation of the classical TF-IDF similarity measure from information retrieval to compare refactored code elements, and a new strategy to compare the similarity of fields by taking into account the similarity of the statements that reads from or writes to them.

\todo{Falar do RefDiff 1.0 e motivar o RefDiff 2.0, dando enfase na questao de ser  multilinguagem}

%In the paper, we also describe in details a study to evaluate the precision and recall of RefDiff and three existing refactoring detection approaches: Refactoring Miner~\cite{fse2016-why-we-refactor}, Refactoring Crawler~\cite{dig2006automated}, and Ref-Finder~\cite{prete2010template,Kim:2010:RefFinder}. In our study, RefDiff achieved precision of 100\% and recall of 88\%, which were the best results among the evaluated approaches.


In summary, the contributions we deliver in this work are:
\begin{itemize}
\item A major extension of our refactoring detection approach proposed in previous work~\cite{msr2017}, which includes improved detection heuristics and a complete redesign of its core to work with a language-independent model.
We provide a publicly available\footnote{RefDiff and all evaluation data are public available in GitHub:\\
\url{https://github.com/aserg-ufmg/RefDiff}} implementation of our approach that supports Java, C and Javascript.
\item An evaluation of the precision and recall of RefDiff using a large scale dataset of refactorings performed in real-world Java open source projects, comparing it with RMiner, a state-of-the-art tool for detecting refactorings in Java. As a byproduct of this evaluation, we also extend the dataset with new refactoring insntances discovered by our tool. 
\item An evaluation of the precision and recall of RefDiff in real-world C and Javascript open source projects.
\end{itemize}

The remainder of this paper is structured as follows. \todo{continuar} %Section~\ref{SecBackground} describes related work, focusing on the three approaches we compare with RefDiff. Section~\ref{SecApproach} presents the proposed approach in details.
%Section~\ref{SecEval} describes how we evaluated RefDiff and discusses the achieved results.
%Section~\ref{SecThreats} discusses threats to validity and we conclude the paper in Section~\ref{SecConclusion}.

\section{Background}
\label{SecBackground}

Empirical studies on refactoring rely on means to identify refactoring activity. Thus, many different techniques have been proposed and employed for this task.
For example, Murphy-Hill~et~al.~\cite{MurphyHill2012} collected refactoring usage data using a framework that monitors user actions in the Eclipse IDE, including calls to refactoring commands.
Negara~et~al.~\cite{negara2013} also used the strategy of instrumenting the IDE to infer refactorings from fine-grained code edits.
Other studies use metadata from version control systems to identify refactoring changes. For example, Ratzinger~et~al.~\cite{ratzinger2008relation} search for a predefined set of terms in commit messages to classify them as refactoring changes. In specific scenarios, a branch may be created exclusively to refactor the code, as reported by Kim et al.~\cite{kim-tse-2014}.
Another strategy is employed by Soares et al.~\cite{soares2010making}. They propose an approach that identify behavior-preserving changes by automatically generating and running test-cases. While their approach is intended to guarantee the correct behavior of a system after refactoring, it may also be employed to classify commits as behavior-preserving.
Moreover, many existing approaches are based on static analysis.
This is the case of the approach proposed by Demeyer et al.~\cite{demeyer2000finding}, which finds refactored elements by observing changes in code metrics.

Static analysis is also frequently used to find differences in the source code~\cite{dig2006automated, weissgerber2006identifying, tsantalis_empiricalstudy,prete2010template,Kim:2010:RefFinder}.
Approaches based on comparing source code differences have the advantage of beeing able to identify each refactoring operation performed. As RefDiff is one of these approaches, it can be directly compared with others within this category. In the next sections, we will describe three of such approaches.


\subsection{Refactoring Miner}

Refactoring Miner is an approach introduced by Tsantalis~et~al.~\cite{tsantalis_empiricalstudy}, that was later extend by Silva~et~al.~\cite{fse2016-why-we-refactor} to mine refactorings in large scale in git repositories. This tool is capable of identifying 14 high-level refactoring types: \emph{Rename Package/Class/Method}, \emph{Move Class/Method/Field}, \emph{Pull Up Method/Field}, \emph{Push Down Method/Field}, \emph{Extract Method}, \emph{Inline Method}, 
and \emph{Extract Superclass/Interface}.

Refactoring Miner runs a lightweight algorithm, similar to the UMLDiff proposed by Xing and Stroulia~\cite{Xing:2005}, for differencing object-oriented models, inferring the set of classes, methods, and fields added, deleted or moved between two code revisions. 
First, the algorithm matches code entities in a top-down order (starting from the classes and going to the methods and fields) looking for exact matches on their names and signatures (in the case of methods).
Next, the removed/added elements between the two models are matched based only on the equality of their names in order to find changes in the signatures of fields and methods.
Third, the removed/added classes are matched based on the similarity of their members at signature level.
Finally, a set of rules enforcing structural constraints is applied to identify specific types of refactorings.

In a first study, using the version histories of JUnit, HTTPCore, and HTTPClient, Tsantalis~et~al.~\cite{tsantalis_empiricalstudy} found 8 false positives for the \emph{Extract Method} refactoring (96.4\% precision) and 4 false positives for the \emph{Rename Class} refactoring (97.6\% precision). No false positives were found for the remaining refactorings.
In a second study that mined refactorings in 285 GitHub hosted Java repositories, Silva~et~al.~\cite{fse2016-why-we-refactor} found 1,030 false positives out of 2,441 refactorings (63\% precision). However, the authors also evaluated Refactoring Miner using as a benchmark the dataset reported by Chaparro~et~al.~\cite{Chaparro:2014}, in which it achieved 93\% precision and 98\% recall.

\todo{Reescrever essa secao levando em conta o paper ICSE do RMiner}

\subsection{Refactoring Crawler}

Refactoring Crawler, proposed by Dig~et~al.~\cite{dig2006automated}, is an approach capable of finding seven high-level refactoring types: \emph{Rename Package/Class/Method}, \emph{Pull Up Method}, \emph{Push Down Method}, \emph{Move Method}, and \emph{Change Method Signature}.
It uses a combination of a syntactic analysis to detect refactoring candidates and a more expensive reference graph analysis to refine the results.

First, Refactoring Crawler analyzes the abstract syntax tree of a program and produces a tree, in which each node represents a source code entity (package, class, method, or field).
Then, it employs a technique known as \emph{shingles encoding} to find 
similar pairs of entities, which are candidates for refactorings.
Shingles are representations for strings with the following property: if a string changes slightly, then its shingles also change slightly.
In a second phase, Refactoring Crawler applies specific strategies for detecting each refactoring type, and computes a more costly metric that determines the similarity of references among code entities in the two versions of the system. For example, two methods are similar if the sets of methods that call them are similar, and the sets of methods they call are also similar.
The strategies to detect refactorings are repeated in a loop until no new refactorings are found. Therefore, the detection of a refactoring, such as a rename, may change the reference graph of code elements and enable the detection of new refactorings.

The authors evaluated Refactoring Crawler comparing pairs of releases of three open source software components: Eclipse UI, Struts, and JHotDraw. Such components were chosen because they provided detailed release notes describing API changes. The authors relied on such information and on manual inspection to build an oracle of known refactorings in those releases, containing 131 refactorings in total.
The reported results are: Eclipse UI (90\% precision and 86\% recall), Struts (100\% precision and 86\% recall), and JHotDraw (100\% precision and 100\% recall).


\subsection{Ref-Finder}

Ref-Finder, proposed by Prete~et~al.~\cite{prete2010template,Kim:2010:RefFinder}, is an approach based on logic programming capable of identifying 63 refactoring types from the Fowler's catalog\cite{Fowler:1999}.
The authors express each refactoring type by defining structural constraints, before and after applying a refactoring to a program, in terms of template logic rules.

First, Ref-Finder traverses the abstract syntax tree of a program and extracts facts about code elements, structural dependencies, and the content of code elements, to represent the program in terms of a database of logic facts. Then, it uses a logic programming engine to infer concrete refactoring instances, by creating a logic query based on the constraints defined for each refactoring type.
The definition of refactoring types also consider ordering dependencies among them. This way, lower-level refactorings may be queried to identify higher-level, composite refactorings.
The detection of some types of refactoring requires a special logic predicate that indicates that the similarity between two methods is above a threshold. For this purpose, the authors implemented a block-level clone detection technique, which removes any beginning and trailing parenthesis, escape characters, white spaces and return keywords and computes word-level similarity between the two texts using the longest common sub-sequence algorithm.

The authors evaluated Ref-Finder in two case studies.
In the first one, they used code examples from the Fowler's catalog to create instances of the 63 refactoring types. The authors reported 93.7\% recall and 97.0\% precision for this first study.
In the second study, the authors used three open-source projects: Carol, jEdit, and Columba. In this case, Ref-Finder was executed in randomly selected pairs of versions. From the 774 refactoring instances found, the authors manually inspected a sample of 344 instances and found that 254 were correct (73.8\% precision).
However, in a study by Soares~et~al.~\cite{Soares:2013} using a set of randomly select versions of JHotDraw and Apache Common Collections containing 81 refactoring instances in total, Ref-Finder achieved only 35\% precision and 24\% recall.

\section{Proposed Approach}

Our approach consists in two phases: Source Code Analysis and Relationship Analysis.
In the first phase, Source Code Analysis, we take as input two revisions of a system, $v_1$ and $v_2$, and build two models that represent their source code.
Both models have the form of a tree, in which each node corresponds to a code element (classes, functions, etc.).
In the second phase, Relationship Analysis, we compute the set $R_{1,2}$, which contains triples of the form $(n_1, n_2, t)$, where $n_1$ is a code element from revision $v_1$, $n_2$ is a code element from revision $v_2$ and $t$ is a relationship type.
Such relationships may denote a high-level refactoring operation (move, rename, extract, etc.) or an exact correspondence between the code elements.
For example, consider the diff between two revisions of a system depicted in Figure~\ref{FigDiff1}.
Among other changes, the class \codeinl{Calculator}, declared in revision~1, is renamed to \codeinl{FpCalculator} in revision~2. This would correspond to a relationship of the type \emph{Rename} between them.
In the next sections we describe in details each phase of our approach.

\begin{figure*}[htb]
\centering
\includegraphics[width=0.9\textwidth]{img/diff1.pdf}
\caption{Illustrative diff between two revisions of a system}
\label{FigDiff1}
\end{figure*}


\subsection{Phase 1: Source Code Analysis}

The goal of this phase is compute a language-independent model that represents the source code of the system, which we denote from now on as \emph{Code Structure Tree} (CST). The CST is a tree-like structure that resembles an \emph{Abstract Syntax Tree} (AST), however, in this representation we are interested in coarse-grained code elements, that encompass a code region and may be referred by an identifier in other parts of the system.

To construct the CST, we need to parse the source code, generate the AST for the target programming language, and extract the necessary information.
Thus, the decision of which types of AST nodes becomes CST nodes depends on language.
For example, in Java we represent classes, enums, interfaces, and methods as CST nodes.
In contrast, local variables are not represented.
Nevertheless, it is important to note that the granularity of the CST nodes determines the granularity of the relationships we are able to find, e.g., we can only find relationships between methods if we represent methods in the CST.
Table~\ref{TabCstNodes} lists the types of AST nodes that are represented in the CST for each programming language we support.

\begin{table}[htbp]
\renewcommand{\arraystretch}{1.2}
\caption{AST nodes that are represented in the CST}
\label{TabCstNodes}
\centering
\begin{tabular}{@{}ll@{}}
\toprule
Language & Node types \\
\midrule
Java & class, enum, interface, and method \\
C & file and function \\
JavaScript & file, class, and function \\
\bottomrule
\end{tabular}
\end{table}

\begin{figure}[htb]
\centering
\includegraphics[width=1.0\linewidth]{img/cstDiff1.pdf}
\caption{CST of both revisions of the example system from Figure~\ref{FigDiff1}}
\label{FigJavaToCst}
\end{figure}

Figure~\ref{FigJavaToCst} exemplifies the transformation of the example system from Figure~\ref{FigDiff1} into a corresponding CST.
In revision~1, the class \codeinl{Main} is declared with a single method \codeinl{main} and the class \codeinl{Calculator} contains two methods: \codeinl{sum} and \codeinl{min}.
Note that all those classes and methods become nodes in the CST for revision~1, preserving the same nesting structure of the source code. Analogously, the figure also depicts the CST for revision 2, which contains seven nodes in total (two classes and five methods).

Besides the representation of the code elements, the CST also holds a simplified call graph and  a type hierarchy graph of the nodes within the CST, that is, there are edges to represent whether a certain node $n_1$ calls $n_2$, or $n_1$ is a subtype of $n_2$. The first information is necessary to find \emph{Extract} and \emph{Inline} relationships between code elements, while the second is used to find inheritance related relationships such as \emph{Pull Up} and \emph{Push Down}.

Moreover, along with each node of the CST, we store the following information:
\begin{description}
    \item[Identifier] \hfill \\
    An identifier of the code element in its declared scope. 
    The identifier is usually the name of the code element, but it may also contain additional information to avoid ambiguities.
    For example, the identifier of the class \codeinl{Calculator} from Figure~\ref{FigJavaToCst} is simply its name, but the identifier of the method \codeinl{sum} is \codeinl{sum(double,double)} because there could be an overloaded method with a different signature.
    
    \item[Namespace] \hfill \\
    An optional prefix that, along with the identifier, globally identifies the code element. 
    This information only applies to root nodes and corresponds to the package or folder that the element is contained. For example, the namespace of the class \codeinl{Calculator} from Figure~\ref{FigJavaToCst} is \codeinl{my.calc.}.
    
    \item[Node type] \hfill \\
    A string that identifies the node type in the target language (class, function, method, etc.).
    
    \item[Parameters list]  \hfill \\
    An optional list of the name of the parameters, in the case the node corresponds to a method or function.
    
    \item[Tokenized source code]  \hfill \\
    The source code of the code element in the form of a list of tokens. 
    This information is necessary to compute the similarity between code elements, as explained in Section~\ref{SecCodeSim}.
    
    \item[Tokenized source code of the body]  \hfill \\
    The source code of the body of the code element in the form of a list of tokens. 
    This information is optional, as not every node possess a body (e.g., abstract members).
    This information is also necessary to compute the similarity between code elements in the special cases of \textit{Extract} and \textit{Inline} relationships, as explained in Section~\ref{SecSimX}.
    
\end{description}

It is worth noting that we generate the CST only for source files that have been added, removed, or modified between revisions. 
Such information can be efficiently obtained from version control systems, without the need to read the content of all files within the repository.
This way, we avoid a costly operation that might compromise the scalability of our approach, as large repositories contains thousands of source files, but only a small fraction of them change between subsequent revisions.

Although the construction of the CST is a language-specific process, from this point on, the approach is language-independent and relies only on the information encoded in the CST.
This way, one is able to extend our approach to work with different programming languages only by implementing the \emph{Source Code Analysis} module.
To demonstrate that capability, we provide implementations for three programming languages: Java, C, and JavaScript.


\subsection{Phase 2: Relationship Analysis}


\begin{table*}[htbp]
\renewcommand{\arraystretch}{1.3}
\caption{Relationship types and the conditions to find them}
\label{TabRelationshipTypes}
\centering

\begin{tabular}{@{}lll@{}}
\toprule
Relationship type & Conditions \\
\midrule
& \multicolumn{2}{l}{$(n_1, n_2) \in N^- \times N^+$, such that:}\\
Same & & $\rdsig(n_1) = \rdsig(n_2) \land \rdparent(n_1)' = \rdparent(n_2) \land \rdtype(n_1) = \rdtype(n_2)$ \\
Convert Type & & $\rdsig(n_1) = \rdsig(n_2) \land \rdparent(n_1)' = \rdparent(n_2) \land \rdtype(n_1) \neq \rdtype(n_2)$ \\
Pull Up & & $\rdsig(n_1) = \rdsig(n_2) \land \rdsub(\rdparent(n_1)', \rdparent(n_2))$ \\
Push Down & & $\rdsig(n_1) = \rdsig(n_2) \land \rdsub(\rdparent(n_2), \rdparent(n_1)')$ \\
Change Signature & & $\rdname(n_1) = \rdname(n_2) \land \rdsig(n_1) \neq \rdsig(n_2) \land \rdparent(n_1)' = \rdparent(n_2) \land \rdsim(n_1, n_2) > \tau$ \\
Move & & $\rdname(n_1) = \rdname(n_2) \land \rdparent(n_1)' \neq \rdparent(n_2) \land \rdsim(n_1, n_2) > \tau$ \\
Rename & & $\rdname(n_1) \neq \rdname(n_2) \land \rdparent(n_1)' = \rdparent(n_2) \land \rdsim(n_1, n_2) > \tau$ \\
Move and Rename & & $\rdname(n_1) \neq \rdname(n_2) \land \rdparent(n_1)' \neq \rdparent(n_2) \land \rdsim(n_1, n_2) > \tau$ \\
\addlinespace
& \multicolumn{2}{l}{$(n_1, n_2) \in N_1^= \times N^+$, such that:}\\
Extract Supertype & & $\exists (n_3, n_4, \mathit{PullUp}) \in R_{1,2}\, (n_1 = \rdparent(n_3) \land n_2 = \rdparent(n_4))$ \\
Extract & & $\rduses(n_1', n_2) \land \rdparent(n_1)' = \rdparent(n_2) \land \rdsimx(n_2, n_1) > \tau$ \\
Extract and Move & & $\rduses(n_1', n_2) \land \rdparent(n_1)' \neq \rdparent(n_2) \land \rdsimx(n_2, n_1) > \tau$ \\
\addlinespace
& \multicolumn{2}{l}{$(n_1, n_2) \in N^- \times N_2^=$, such that:}\\
Inline & & $\rduses(n_1, n_2') \land \rdsimx(n_1, n_2) > \tau$ \\
\bottomrule
\end{tabular}

\vspace{1em}
\begin{tabular}{@{}lll@{}}
\midrule
\multicolumn{3}{c}{\textbf{Definitions}}\\
\begin{tabular}{@{}ll@{}}
$N_1^=$ & the set of nodes from $N_1$ that matches with another node from $N_2$\\
$N_2^=$ & the set of nodes from $N_2$ that matches with another node from $N_1$\\
$N^-$ & the set of presumably deleted nodes ($N_1 \setminus N_1^=$)\\
$N^+$ & the set of presumably added nodes ($N_2 \setminus N_2^=$)\\
$n'$ & the code element that matches with $n$ in the other revision\\
$\rdparent(n)$ & parent of a node $n$ (it may be a namespace or another CST node)\\
\end{tabular}
& &
\begin{tabular}{@{}ll@{}}
$\rdname(n)$ & simple name of the code element $n$\\
$\rdsig(n)$ & identifier of the code element $n$\\
$\rdtype(n)$ & node type of the code element $n$\\
$\rdsub(n_1, n_2)$ & $n_1$ is subtype of $n_2$\\
$\rduses(n_1, n_2)$ & $n_1$ uses $n_2$\\
$\rdsim(n_1, n_2)$ & similarity between $n_1$ and $n_2$\\
$\rdsimx(n_1, n_2)$ & extract similarity between $n_1$ and $n_2$\\
\end{tabular}
\\
\midrule
\end{tabular}

\end{table*}



This phase takes as input the CST's of revisions $v_1$ and $v_2$ and outputs the set of relationships $R_{1,2}$. Let $N_1$ and $N_2$ be the sets of code elements from the CST's of $v_1$ and $v_2$ respectively, each relationship $r \in R_{1,2}$ is a triple $(n_1, n_2, t)$, where $n_1 \in N_1$, $n_2 \in N_2$, and $t$ is a relationship type. The types of relationships are listed in the first column of Table~\ref{TabRelationshipTypes}, and can be subdivided into two groups:
\begin{enumerate}
\item \textbf{Matching relationships}, which include \textit{Same}, \textit{Convert Type}, \textit{Pull Up}, \textit{Push Down}, \textit{Change Signature}, \textit{Move}, \textit{Rename}, and \textit{Move and Rename}.
Those relationships indicates that the node $n_1$ corresponds to $n_2$ in the subsequent revision.
We say that a node $n_1$ matches with $n_2$ if exists a relationship $(n_1, n_2, t) \in R_{1,2}$ such that $t$ is a matching relationship.

\item \textbf{Non-matching relationships}, which include \textit{Extract Supertype}, \textit{Extract}, \textit{Extract and Move}, and \textit{Inline}.
Those relationships don't indicate a matching, but rather that either node $n_1$ is decomposed to create $n_2$, or node $n_1$ is incorporated into $n_2$.
\end{enumerate}


\subsubsection{General procedure to find relationships}

Our approach employs an iterative procedure to find the relationships.
Let $N^-$ be the set of deleted nodes between revisions and $N^+$ be the set of added nodes between revisions.
Initially, we define $R_{1,2} \gets \emptyset$, $N^- \gets N_1$, and $N^+ \gets N_2$, that is, before finding any relationship, we presume that all code elements from $v_1$ were deleted and  all code elements from $v_2$ were add.
Then, in each step, we look for pairs $(n_1, n_2)$ that match the criteria for a specific relationship type described in Table~\ref{TabRelationshipTypes}, and add the corresponding relationships to $R_{1,2}$.
Additionally, we remove $n_1$ from $N^-$ and $n_2$ from $N^+$, in the case of a matching relationship.
By the end of this procedure, $R_{1,2}$ contains all relationships found, $N^-$ contains the removed nodes, and $N^+$ contains the added nodes.
Specifically, we follow the steps below to look for relationships:
\begin{enumerate}

\item We look for \textit{Same} and \textit{Convert type} relationships in a top-down order. The conditions for those relationships (see Table~\ref{TabRelationshipTypes}) relies on matching by the identifier and parent node, i.e., code elements with the same identifier and parent node are assumed to be the same.
This way, many code elements are usually matched at this step, reducing the number of possibilities that need to be checked in the next steps.

\item We look for \textit{Change signature}, \textit{Move}, \textit{Rename}, and \textit{Move and Rename} relationships.
The conditions to find those relationships relies on a code similarity metric, which is described in details in Section~\ref{SecCodeSim}.
In this step, we limit our search to top-level nodes.
The rationale for this decision is explained in Section~\ref{SecDependentConflictingRel}.

\item We look for \textit{Pull Up} and \textit{Push Down} relationships.

\item Once again, we look for \textit{Change signature}, \textit{Move}, \textit{Rename}, and \textit{Move and Rename} relationships, but this time we consider both parent and children nodes.

\item We look for \textit{Extract Supertype} relationships.

\item We look for \textit{Extract} and \textit{Extract and Move} relationships.

\item Finally, we look for \textit{Inline} relationships.

\end{enumerate}


\subsubsection{Dependent and conflicting relationship conditions}
\label{SecDependentConflictingRel}

In some cases, correctly finding a relationship depends on finding a prior relationship.
Specifically, the conditions from Table~\ref{TabRelationshipTypes} that contains the expression $\rdparent(n_1)'$, which denotes the matching element of the parent of $n_1$, may yield different results depending on the current elements of $R$.
For example, consider 



The rationale is that matching a child node before matching its parent may lead to incorrect relationship types. For example, suppose that a class \codeinl{X} with a method \codeinl{m} is renamed to \codeinl{Y} in the subsequent revision, but method \codeinl{m} is unchanged. If we match method \codeinl{m} before matching class \codeinl{X}, we would incorrectly assume that \codeinl{m} moved from class \codeinl{X} to class \codeinl{Y}, but actually \codeinl{X} was renamed and \codeinl{m} is the same.

\todo{write}


\subsection{Code Similarity}
\label{SecCodeSim}

\begin{figure*}[htb]
\renewcommand{\arraystretch}{1.3}
\centering
\footnotesize
\begin{tabular}{@{}llll@{}}
\begin{tabular}{p{6.5cm}}
\multicolumn{1}{c}{\textbf{Source code of a class}} \\
\begin{lstlisting}
public class Calculator {

  public int sum(int x, int y) {
    return x + y;
  }

  public int min(int x, int y) {
    if (x < y) return x;
    else return y;
  }

  public double power(int b, int e) {
    return Math.pow(b, e);
  }
}
\end{lstlisting}\\

\end{tabular}
& {\Large $\Rightarrow$} &
\begin{tabular}{|l|r|r|r|}
\multicolumn{4}{c}{\textbf{Multiset of tokens for each method}} \\
\hline
Token $t$ & $m_{\mathtt{sum}}(t)$ & $m_{\mathtt{min}}(t)$ & $m_{\mathtt{power}}(t)$\\
\hline
\codeinl{return} & 1 & 2 & 1 \\
\codeinl{x}      & 1 & 2 & 0 \\
\codeinl{+}      & 1 & 0 & 0 \\
\codeinl{y}      & 1 & 2 & 0 \\
\codeinl{;}      & 1 & 2 & 1 \\
\codeinl{if}     & 0 & 1 & 0 \\
\codeinl{(}      & 0 & 1 & 1 \\
\codeinl{<}      & 0 & 1 & 0 \\
\codeinl{)}      & 0 & 1 & 1 \\
\codeinl{else}   & 0 & 1 & 0 \\
\codeinl{Math}   & 0 & 0 & 1 \\
\codeinl{.}      & 0 & 0 & 1 \\
\codeinl{pow}    & 0 & 0 & 1 \\
\codeinl{b}      & 0 & 0 & 1 \\
\codeinl{,}      & 0 & 0 & 1 \\
\codeinl{e}      & 0 & 0 & 1 \\
\hline
\end{tabular} &
\begin{tabular}{|r|}
\multicolumn{1}{c}{} \\
\hline
$n_t$\\
\hline
3 \\
2 \\
1 \\
2 \\
3 \\
1 \\
2 \\
1 \\
2 \\
1 \\
1 \\
1 \\
1 \\
1 \\
1 \\
1 \\
\hline
\end{tabular}
\end{tabular}
\caption{Transformation of the body of each method into a multiset of tokens}
\label{FigSourceCodeTransformation}
\end{figure*}

A key element of our approach to find relationships, as mentioned previously, is computing the similarity.
The first step to compute the similarity between code elements is to represent their source code as a multiset (or bag) of tokens.
A multiset is a generalization of the concept of a set, but it allows multiple instances of the same element.
The multiplicity of an element is the number of occurrences of that element within the multiset. Formally, a multiset can be defined in terms of a multiplicity function $m: U \to \mathbb{N}$, where $U$ is the set of all possible elements. In other words, $m(t)$ is the multiplicity of the element $t$ in the multiset. Note that the multiplicity of an element that is not in the multiset is zero.

For example, Figure~\ref{FigSourceCodeTransformation} depicts the transformation of the source code of three methods (\codeinl{sum}, \codeinl{min}, and \codeinl{power}), of the class \codeinl{Calculator}, into multisets of tokens. In this figure, the multiplicity function $m$ for each method is represented in a tabular form. For example, the multiplicity of the token \codeinl{y} in method \codeinl{min} is two (i.e., $m_{\mathtt{min}}(\mathtt{y}) = 2$), whilst the multiplicity of the token \codeinl{if} in method \codeinl{power} is zero (i.e., $m_{\mathtt{power}}(\mathtt{if}) = 0$).


Later, to compute the similarity between two code elements $e_1$ and $e_2$, we use a generalization of the Jaccard coefficient, known as weighted Jaccard coefficient~\cite{chierichetti2010finding}.
Let $U$ be the set of all possible tokens and $w(e, t)$ be a weight function of a token $t$ for the entity $e$.
We define the similarity between $e_1$ and $e_2$ by the following formula:


\begin{align}
\rdsim(e_1, e_2) = \frac{\sum_{t \in U} \min(w(e_1, t), w(e_2, t))}
                        {\sum_{t \in U} \max(w(e_1, t), w(e_2, t))}
\end{align}

%\subsubsubsection{Weight of a token for a code entity}

Our similarity function is based on a weighting function $w(e, t)$ that expresses the importance a token $t$ for a code entity $e$.
In fact, some tokens are more important than others to discriminate a code element.
For example, in Figure~\ref{FigSourceCodeTransformation}, all three methods contain the token \codeinl{return}. In contrast, only one method (\codeinl{power}) contains the token \codeinl{Math}. Therefore, the later is a better indicator of similarity between methods than the former.

In order to take this into account, we employ a variation of the TF-IDF weighting scheme~\cite{salton1986introduction}, which is a well-known technique from information retrieval.
TF-IDF, which is the short form of \emph{Term Frequency–Inverse Document Frequency}, reflects how important a term is to a document within a collection of documents.
In the context of code elements, we consider a token as a term, and the body of a method (or class) as a document.

Let $E$ be the set of all code elements and $n_t$ be the number of elements in $E$ that contains the token $t$,
we define the weight of $t$ for a code element $e$ as the function $w(e, t)$, which is defined by the following formula:
\begin{align}
w(e, t) = m_e(t) \times \mathit{idf}(t)
\end{align}

\noindent where $m_e(t)$ is the multiplicity of $t$ in $e$, and $\mathit{idf}(t)$ is the Inverse Document Frequency, which is defined as:
\begin{align}
\mathit{idf}(t) = \log (1 + \frac{|E|}{n_t})
\end{align}

Note that the value of $\mathit{idf}(t)$ decreases as $n_t$ increases, because the more frequent a token is among the collection of code elements, the less important it is to distinguish code elements.
For example, in Figure~\ref{FigSourceCodeTransformation}, the token \codeinl{y} occurs in two methods (\codeinl{sum} and \codeinl{min}). Thus, its $\mathit{idf}$ is:

\[
\mathit{idf}(\mathtt{y}) = 
\log (1 + \frac{|E|}{n_t}) = 
\log (1 + \frac{3}{2}) = 0.398
\]

On the other hand, the token \codeinl{else} occurs in one method ($\mathtt{min}$), and its $\mathit{idf}$ is:

\[
\mathit{idf}(\mathtt{else}) = 
\log (1 + \frac{|E|}{n_t}) = 
\log (1 + \frac{3}{1}) = 0.602
\]


\subsubsection{Extract and Inline Similarity}
\label{SecSimX}

While the similarity function presented previously is suitable to compute whether two code elements are are similar, in some situations we need to assess whether the source code of an element is contained within another one. This is the case of \emph{Extract}, and \emph{Inline} relationships. For example, if a method $e_2$ is extracted from $e_1$, the source code of $e_2$ should be contained within $e_1$ prior to the refactoring, although $e_1$ and $e_2$ may be significantly different from each other. Analogously, if a method $e_1$ is inlined into $e_2$, the source code of $e_1$ should be contained within $e_2$.
Thus, we defined a specialized version of the similarity function $\rdsimx$ defined by the following formula:
\begin{align}
\rdsimx(e_1, e_2) = \frac{\sum_{t \in U} \min(w(e_1, t), w(e_2, t))}
                        {\sum_{t \in U} w(e_1, t)}
\end{align}

The rationale behind this formula is that the similarity is at maximum when the multiset of tokens of $e_1$ is a subset of the multiset of tokens of $e_2$, i.e., all tokens from $e_1$ can be found in $e_2$. Note that, given this definition, $\rdsimx(e_1, e_2) \neq \rdsimx(e_2, e_1)$.



\section{Evaluation with Java Projects}
\label{sec:eval:java}

In this section, we evaluate the precision and recall of our approach using a recently proposed dataset of refactorings performed in real-world Java open source projects. We also compare RefDiff's accuracy with RMiner, 
a state-of-the-art tool for detecting refactorings in Java.
First, we present our evaluation design (Section~\ref{sec:eval:java:design}) and then we present the results (Section~\ref{sec:eval:java:results}).

\subsection{Evaluation Design}
\label{sec:eval:java:design}

To evaluate the precision and recall of RefDiff in Java we initially use an oracle proposed by Tsantalis et al.~\cite{tsantalis2018rminer}.
This oracle includes 3,188 manually-validated refactoring instances, detected in 538 commits from 185 open source projects, and covering 15 refactoring operations.
We also use this oracle to compare RefDiff's precision and recall against RMiner.
For the purpose of the comparison, we restricted the oracle to 11 refactoring types supported by both tools.
Specifically, we excluded \emph{Change Package}, \emph{Move Field}, \emph{Push Down Field} and \emph{Pull Up Field} from the analysis as they are not supported by RefDiff.
Moreover, \emph{Convert Type} and \emph{Change Signature}, although supported by RefDiff, are not evaluated because they are not covered by the oracle.
In total, our modified oracle contains 3,031 confirmed refactoring instances.
Additionally, it also contains 704 refactoring instances classified as false positives in the process of manual validation performed by Tsantalis et al.~\cite{tsantalis2018rminer}.
%Last, it is worth noting that the \emph{Rename Class} category also includes instances of \emph{Move and Rename Class}. Similarly, the \emph{Extract Method} category also includes instances of \emph{Extract and Move Method}. This measure was necessary because the oracle did not distinguish between those refactorings reliably.

First, we run RefDiff on each commit of the oracle. For each detected refactoring $r$ we checked whether $r$ is in the oracle, which may yield three outcomes: (i) if $r$ is a confirmed refactoring from the oracle, then it is a true positive; (ii) if $r$ is a false refactoring from the oracle, then it is a false positive; (iii) otherwise, $r$ was inspected by two authors of this paper to assess whether it is a false positive or a true positive not covered by the oracle.
This extra manual validation is needed because the initial oracle must not be granted as complete, i.e., including all refactorings performed in the set of analysed commits.
Specifically, this oracle was constructed using a triangulation approach, based on an initial list of refactorings produced by RMiner and RefDiff 1.0. For this reason, it might miss true refactorings only detected by the improved implementation of RefDiff, described in this paper.

\begin{figure*}[!t]
\centering
\includegraphics[width=0.8\textwidth]{img/diff2.pdf}
\caption{Illustrative diff of an \emph{Inline Method} refactoring considered as true positive by the validators.}
\label{FigDiff2}
\end{figure*}

After following this procedure, RefDiff 2.0 detected 278 new refactoring instances (i.e., not listed in the initial oracle), which were  validated by two paper's authors, called here validators. In the case of 173 refactorings (62\%), the validators agreed on their classification, including 115 refactorings labelled as true positives by both validators and 58 labelled as false positives. After this initial and independent validation, the validators discussed together the remaining 105 cases (38\%), to reach an agreement. As a result, 66 refactorings were considered true positives and 39 refactorings were classified as false positives. Figure~\ref{FigDiff2} shows a first example of true positive. RefDiff identified an \emph{Inline Method} refactoring, consisting on the substitution of the method named \codeinl{acquireNodeRelationshipCursor}, which was removed in the same commit, by invocations to methods \codeinl{get()} and \codeinl{init()}. We emphasize in this figure the changes that correspond to this refactoring. This indication of refactoring was not listed in the oracle. The validators individually analyzed this commit and both agreed that an \emph{Inline Method} was applied in this commit.



Figure~\ref{FigDiff3} shows another example of true positive, this time as an indication of \emph{Extract} and \emph{Move Method} refactorings. Developers removed the invocation to the method \codeinl{readValue(String, Class)}, which was extracted and moved to the class \codeinl{Controls}. An invocation to this new method \codeinl{validateControlsString(String)} was replaced in the same line. Similarly to the example in Figure~\ref{FigDiff2}, both validators agreed that \emph{Extract} and \emph{Move Method} were applied in this commit.

\begin{figure*}[htb]
\centering
\includegraphics[width=0.8\textwidth]{img/diff3.pdf}
\caption{Illustrative diff of an \emph{Extract Method} refactoring considered as true positive by the validators.}
\label{FigDiff3}
\end{figure*}

After the manual validation of the refactoring instances it was possible to identify some common causes of failure.
In particular, one common reason for \emph{Move Method} false positives was missed \emph{Move/Rename Class}, i.e., RefDiff did not detect that the entire class has been moved (or renamed), and incorrectly reported that several of its members were moved.
We mitigated this issue by introducing a specific heuristic for this case (see Section~\ref{AlgoGeneral}).
Moreover, the analysis of the incorrect reports allowed us to identify and fix a couple of bugs in our implementation.
After applying the aforementioned fixes to RefDiff, it was necessary to run it again and compare the results against the update oracle. This time, 51 new refactoring instances were found, which were  validated by one of the authors. 24 of them (47\%) were classified as true positive, whilst 27 of them (53\%) were classified as false positive.

In summary, after these two rounds of manual validation, 184 new refactorings instances were classified as true positives and therefore included in the oracle.
The expanded oracle includes 3,261 refactoring instances (5.98\% more than the initial one) and it is publicly available at RefDiff's GitHub repository.\footnote{\url{https://github.com/aserg-ufmg/RefDiff}}

\subsection{Results}
\label{sec:eval:java:results}

\begin{table*}[htbp]
\renewcommand{\arraystretch}{1.2}
\caption{Java precision and recall results}
\label{TabResultJava}
\centering
\begin{tabular}{@{}lrlrrrlllrrrll@{}}
\toprule
 & & & \multicolumn{5}{c}{RefDiff} & & \multicolumn{5}{c}{Refactoring Miner}\\
\cmidrule{4-8} \cmidrule{10-14}
Refactoring Type & \# & & TP & FP & FN & Precision & Recall & & TP & FP & FN & Precision & Recall \\
\midrule
Move Class & 1121 & & 1067 & 1 & 54 & \xbar{0.999} & \xbar{0.952} & & 1017 & 0 & 104 & \xbar{1.000} & \xbar{0.907} \\
Move Method & 320 & & 256 & 38 & 64 & \xbar{0.871} & \xbar{0.800} & & 210 & 10 & 110 & \xbar{0.955} & \xbar{0.656} \\
Move and Rename/Rename Class & 92 & & 80 & 10 & 12 & \xbar{0.889} & \xbar{0.870} & & 59 & 1 & 33 & \xbar{0.983} & \xbar{0.641} \\
Rename Method & 346 & & 238 & 19 & 108 & \xbar{0.926} & \xbar{0.688} & & 270 & 6 & 76 & \xbar{0.978} & \xbar{0.780} \\
Extract Interface & 24 & & 21 & 3 & 3 & \xbar{0.875} & \xbar{0.875} & & 20 & 0 & 4 & \xbar{1.000} & \xbar{0.833} \\
Extract Superclass & 70 & & 52 & 0 & 18 & \xbar{1.000} & \xbar{0.743} & & 68 & 3 & 2 & \xbar{0.958} & \xbar{0.971} \\
Pull Up Method & 91 & & 75 & 2 & 16 & \xbar{0.974} & \xbar{0.824} & & 72 & 0 & 19 & \xbar{1.000} & \xbar{0.791} \\
Push Down Method & 40 & & 38 & 2 & 2 & \xbar{0.950} & \xbar{0.950} & & 33 & 0 & 7 & \xbar{1.000} & \xbar{0.825} \\
Extract/Extract and Move Method & 1037 & & 686 & 29 & 351 & \xbar{0.959} & \xbar{0.662} & & 796 & 12 & 241 & \xbar{0.985} & \xbar{0.768} \\
Inline Method & 120 & & 86 & 6 & 34 & \xbar{0.935} & \xbar{0.717} & & 97 & 1 & 23 & \xbar{0.990} & \xbar{0.808} \\
\addlinespace
Total & 3261 & & 2599 & 110 & 662 & \xbar{0.959} & \xbar{0.797} & & 2642 & 33 & 619 & \xbar{0.988} & \xbar{0.810} \\
\bottomrule
\end{tabular}
\end{table*}

Table~\ref{TabResultJava} shows the precision and recall results for RefDiff and RMiner using the oracle described in the previous section. The overall precision and recall of RefDiff is 95.9\% and 79.7\%, respectively.
Precision ranges from 87.1\% (\emph{Move Method}) to 100.0\% (\emph{Extract Superclass}), and it is above 90\% for 7 of the 10 refactoring types.
Recall ranges from 66.2\% (\emph{Extract Method}) to 95.2\% (\emph{Move Class}), and it is above 80\% for 6 of the 10 refactoring types.
On its turn, RMiner achieves 98.8\% of overall precision (ranging from 95.5\% to 100.0\%) and 81.0\% of overall recall (ranging from 64.1\% to 97.1\%).
When we analyze individual refactoring types, RefDiff's precision is always lower, but recall is higher in 6 refactoring types.
Thus, we conclude that both tools have very similar recall, but RMiner's precision is consistently higher.
Nevertheless, we argue that RefDiff's precision of 95.9\% is satisfactory, specially when we consider that our approach is programming language neutral.






\section{Evaluation with JavaScript and C}
\label{sec:eval:js:c}

Beyond the Java evaluation, we also evaluated precision and recall of RefDiff in JavaScript and C. Unfortunately, we did not find a dataset with detailed information about real refactorings performed in these languages that we could use as an oracle.
Therefore, we had to adopt a different strategy. 
%We also evaluate RefDiff with refactorings performed in two important but very different programming languages: JavaScript (a widely popular dynamic programming language, used mostly to build web applications) and C (a procedural programming language, used mostly to implement system software).
%In the literature, we did not find a dataset with detailed information about real refactorings performed in these languages. Therefore, 
To evaluate precision, we manually validated the refactorings detected by RefDiff in a set of programs, in both languages (Section~\ref{sec:eval:js:c:precision}). Then, to evaluate recall, we created an oracle of manually validated refactoring operations performed in another set of programs (Section~\ref{sec:eval:js:c:recall}).  After that, in Section~\ref{sec:eval:js:c:results}, we report the precision and recall achieved by RefDiff.  We are not aware of any other tool for detecting refactorings in these languages. Therefore, in this second evaluation, it was not possible to compare RefDiff's results with competitor tools.

%\subsection{Evaluation Design}
%\label{sec:eval:js:c:design}

%In this section, we defined the steps we followed to compute RefDiff's precision (Section~\ref{sec:eval:js:c:precision})
%and recall (Section~\ref{sec:eval:js:c:recall}), when used to detect refactorings in JavaScript and C commits.

\subsection{Evaluation Design: Precision}
\label{sec:eval:js:c:precision}


To compute RefDiff's precision when detecting refactorings in JavaScript and C, we followed these steps:

\begin{enumerate}  
\item We selected the 20 most popular GitHub repositories of each language. For this, we queried the GitHub API for repositories, sorting by stars count---which is a reliable indicator of popularity in GitHub~\cite{icsme2016,jss-2018-github-stars} ---and filtering by programming language.
The resulting list of repositories was manually inspected to discard the ones that are not actual software projects, e.g., tutorials or code samples. Then, we forked each selected repository, to preserve their version histories from future changes pushed to the original project. Table xx shows the name, short description, and number of commits of each selected repository, both for JavaScrpit and C.  \todo{tabela que acabou de ser mencionada}

\item We ran RefDiff in the version history of each repository. To select the commits, we navigate the commit graph backwards, starting from the most recent commit in the master branch. We also discarded merge commits, i.e.,~commits which have two predecessors. The rationale is that comparing a merge commit with their predecessors results in duplicated reports of refactorings applied in the commits prior to the merge operation. Moreover, to avoid over-representing projects with longer histories, we established a limit of 500 commits per repository. For each selected commit, we compared its source code with its predecessor using RefDiff, to detect refactoring operations.

\item Given the list of refactorings detected by RefDiff, we randomly selected 10 instances of each refactoring type to manually assess whether they correspond to actual refactorings (true positives), or incorrect reports (false positives).
When applying the random selection, we enforced the constraint that we should not select two refactoring instances performed in the same commit.
In this way, we avoid selecting similar refactorings which were performed in batch, e.g., multiple classes or functions moved together.
To confirm each refactoring instance, one of the authors manually inspected the diff of the code changes in the corresponding commit.
\end{enumerate}

After following the three steps, we compute precision as $P = \mathit{TP} / (\mathit{TP} + \mathit{FP})$, where $\mathit{TP}$ is the number of true positives and $\mathit{FP}$ is the number of false positives.



\subsection{Evaluation Design: Recall}
\label{sec:eval:js:c:recall}

To compute RefDiff's recall when detecting refactorings in JavaScript and C, we followed three steps:

\begin{enumerate}  
\item We used GitHub API to find refactorings documented in commits from the repositories selected for evaluating precision (Section~\ref{sec:eval:js:c:precision}). Such queries consist in searching for keywords denoting a particular type of refactoring in the commit message, as described in Table xx.\todo{acrescentar esta tabela} For example, when looking for \emph{Rename Function} refactoring instances in JavaScript, we built a query that looks for commits which contain the keywords \texttt{rename} and \texttt{function} in their messages.

\item Given the results of a query, we manually inspected the commit message and the diff of the source code to confirm the applied refactoring. We repeated this procedure until we found 10 instances of each refactoring type or when there were no more results to inspect (which happened only for xxx \todo{detalhar refactorings com menos de 10 instances)}. Each confirmed refactoring was recorded in a normalized textual format compatible with the output of RefDiff.

%It is worth noting that, during this procedure, most of the results were discarded because the commit message did not actually documented a refactoring. Besides, the diff of some of the commits were so large that could not be displayed in the user interface. In summary, we only registered the refactoring instances that we could confirm both in the commit message and in the code diff.

\item We ran RefDiff in the commits that contain documented and manually-validated refactorings to assess whether they are reported (true positives) or missed (false negatives). 

\end{enumerate}

After following these steps, we compute recall as $R = \mathit{TP} / (\mathit{TP} + \mathit{FN})$, where $\mathit{TP}$ is the number of true positives and $\mathit{FN}$ is the number of false negatives.


\subsection{Results}
\label{sec:eval:js:c:results}

In this section, we present the precision and recall results for JavaScript (Section~\ref{sec:eval:js:c:results:js}) and C (Section xx).

\subsubsection{JavaScript Results}
\label{sec:eval:js:c:results:js}

Table~\ref{TabResultJsPrecison} shows the precision results for JavaScript. The overall precision is 91\%. There are three refactorings with precision of 80\%: Rename Function, Move and Rename Function, and Inline Function. For the remaining refactoring types, RefDiff has as precision of 90\% (two refactoring types) or a precision of 100\% (five refactoring types). Table~\ref{TabResultJsRecall} shows the recall results, which reach 88\% when all refactoring types are considered together.
Inline function has the lowest recall (40\%); however, in our oracle has only five instances of this operation. The two most common refactorings (Move File and Move Function, both with 10 instances) have a recall of 100\%.

%The following GitHub repositories were selected to run evaluation procedure:
%\textsc{facebook/\-react}, 
%\textsc{vuejs/\-vue}, 
%\textsc{d3/\-d3}, 
%\textsc{face\-book/\-react-native}, 
%\textsc{angular/\-angular.js}, 
%\textsc{face\-book/\-create-react-app}, 
%\textsc{jquery/\-jquery}, 
%\textsc{atom/\-atom}, 
%\textsc{axios/\-axios}, 
%\textsc{mrdoob/\-three.js}, 
%\textsc{socketio/\-socket.io}, 
%\textsc{reduxjs/\-redux}, 
%\textsc{webpack/\-webpack}, 
%\textsc{Semantic-Org/\-Semantic-UI}, 
%\textsc{hakimel/\-reveal.js}, 
%\textsc{meteor/\-meteor}, 
%\textsc{expressjs/\-express}, 
%\textsc{mui-org/\-material-ui}, 
%\textsc{chartjs/\-Chart.js}


\begin{table}[htbp]
\renewcommand{\arraystretch}{1.2}
\caption{JavaScript precision results}
\label{TabResultJsPrecison}
\centering
\begin{tabular}{@{}lrrl@{}}
\toprule
Refactoring Type & TP & FP & Precision\\
\midrule
Move File & 10 & 0 & \xbar{1.00} \\
Move Class & 2 & 0 & \xbar{1.00} \\
Move Function & 9 & 1 & \xbar{0.90} \\
Rename File & 10 & 0 & \xbar{1.00} \\
Rename Class & 5 & 0 & \xbar{1.00} \\
Rename Function & 8 & 2 & \xbar{0.80} \\
Move and Rename File & 10 & 0 & \xbar{1.00} \\
Move and Rename Function & 8 & 2 & \xbar{0.80} \\
Extract Function & 9 & 1 & \xbar{0.90} \\
Inline Function & 8 & 2 & \xbar{0.80} \\
\addlinespace
Total & 79 & 8 & \xbar{0.91} \\
\bottomrule
\end{tabular}
\end{table}

\begin{table}[htbp]
\renewcommand{\arraystretch}{1.2}
\caption{JavaScript recall results}
\label{TabResultJsRecall}
\centering
\begin{tabular}{@{}lrrl@{}}
\toprule
Refactoring Type & TP & FN & Recall\\
\midrule
Move File & 10 & 0 & \xbar{1.00} \\
Move Function & 10 & 0 & \xbar{1.00} \\
Rename File & 8 & 2 & \xbar{0.80} \\
Rename Function & 9 & 1 & \xbar{0.90} \\
Move and Rename File & 3 & 0 & \xbar{1.00} \\
Move and Rename Function & 6 & 1 & \xbar{0.86} \\
Extract Function & 9 & 1 & \xbar{0.90} \\
Inline Function & 2 & 3 & \xbar{0.40} \\
\addlinespace
Total & 57 & 8 & \xbar{0.88} \\
\bottomrule
\end{tabular}
\end{table}


\subsubsection{C Results}

The following GitHub repositories were selected to run evaluation procedure:
\textsc{torvalds/\-linux}, 
\textsc{firehol/\-netdata}, 
\textsc{antirez/\-redis}, 
\textsc{git/\-git}, 
\textsc{Bilibili/\-ijkplayer}, 
\textsc{php/\-php-src}, 
\textsc{wg/\-wrk}, 
\textsc{ggreer/\-the\_silver\_searcher}, 
\textsc{kripken/\-emscripten}, 
\textsc{vim/\-vim}, 
\textsc{stedolan/\-jq}, 
\textsc{FFmpeg/\-FFmpeg}, 
\textsc{tmux/\-tmux}, 
\textsc{vurtun/\-nuklear}, 
\textsc{obsproject/\-obs-studio}, 
\textsc{libuv/\-libuv}, 
\textsc{swoole/\-swoole-src}, 
\textsc{curl/\-curl}, 
\textsc{irungentoo/\-toxcore}, 
\textsc{pjreddie/\-darknet} 


\begin{table}[htbp]
\renewcommand{\arraystretch}{1.2}
\caption{C precision results}
\label{TabResultCPrecison}
\centering
\begin{tabular}{@{}lrrl@{}}
\toprule
Refactoring Type & TP & FP & Precision\\
\midrule
Move File & 10 & 0 & \xbar{1.00} \\
Move Function & 8 & 2 & \xbar{0.80} \\
Rename File & 10 & 0 & \xbar{1.00} \\
Rename Function & 9 & 1 & \xbar{0.90} \\
Move and Rename Function & 8 & 2 & \xbar{0.80} \\
Change Signature & 10 & 0 & \xbar{1.00} \\
Extract Function & 10 & 0 & \xbar{1.00} \\
Inline Function & 5 & 5 & \xbar{0.50} \\
\addlinespace
Total & 70 & 10 & \xbar{0.88} \\
\bottomrule
\end{tabular}
\end{table}


\begin{table}[htbp]
\renewcommand{\arraystretch}{1.2}
\caption{C recall results}
\label{TabResultCRecall}
\centering
\begin{tabular}{@{}lrrl@{}}
\toprule
Refactoring Type & TP & FN & Recall\\
\midrule
Change Signature & 9 & 1 & \xbar{0.90} \\
Extract Function & 7 & 3 & \xbar{0.70} \\
Inline Function & 9 & 1 & \xbar{0.90} \\
Move File & 10 & 0 & \xbar{1.00} \\
Move Function & 8 & 2 & \xbar{0.80} \\
Move and Rename File & 10 & 0 & \xbar{1.00} \\
Move and Rename Function & 9 & 1 & \xbar{0.90} \\
Rename File & 10 & 0 & \xbar{1.00} \\
Rename Function & 10 & 0 & \xbar{1.00} \\
\addlinespace
Total & 82 & 8 & \xbar{0.91} \\
\bottomrule
\end{tabular}
\end{table}

\section{Challenges and limitations}
\label{sec:challenges}

%In order to support multiple languages, RefDiff distinguish itself from other approaches in at least two main aspects.
%First, our algorithm relies on the CST and on a set of generalized heuristics to abstract away particularities of programming languages.
%Second, our code similarity function is based on tokenized source code and TF-IDF.
%These are key design decision to make RefDiff programming language-agnostic, but they also impose an important limitation.

\textbf{Low-level refactorings:} RefDiff does not detect local refactorings, such as rename, extract or inline a local variable, because the syntactical structure of the source code within a CST node is not represented.
While it is theoretically possible to extend the CST to include finer-grained code elements such as statements, local variables, and others, this would also make it harder to port RefDiff to other programming languages.
%For example, RMiner, which is a Java-based approach, heavily relies on a statement matching algorithm.
%However, not every programming language is based on statements. In fact, functional programming languages such as Haskell relies on expressions rather than statements.
%In essence, by making fewer assumptions about the syntax of the target language we facilitate multi-language support at the price of missing low-level refactorings.

%For example, most of RMiner's heuristics relies on a statement matching algorithm.
%However, not every programming language is statement-based. In fact, functional programming languages such as Haskell relies on expressions rather than statements.
%In summary, RefDiff makes fewer assumptions about the syntax of the target language by relying on the CST and tokenized code, which enables its multi-language capability.


%The CST does not assume a particular hierarchical structure between different types of nodes.
%For instance, it is able to represent methods inside classes, or functions inside functions with arbitrary levels of nesting (e.g., JavaScript).

\noindent\textbf{Generating call graphs:} In our modular architecture, the generation of the CST, which includes information from a call graph and a type hierarchy graph, is delegated to a language-specific plugin.
For languages such as Java, there are reliable parsers and static analyzers that aid in this task (e.g., Eclipse JDT).
However, we acknowledge that generating precise call graphs for untyped languages, such as JavaScript, might be a challenging problem.
Nevertheless, we provided evidences that our approach works well even when the information encoded in the CST is not completely precise.
For example, in our JavaScript implementation---which contains only 689 lines of code in total---we used a simple strategy in which we assume a node $n_1$ uses $n_2$ if $n_1$ contains a function call with the same identifier as $n_2$ and both are defined in the same file.
However, to detect a \emph{Extract} relationship between $n_1$ and $n_2$, we need two other conditions: (i) $n_2$ should be a new method and (ii) the body of $n_2$ should be similar to the code removed from $n_1$ between revisions.
In other words, an incorrect edge in the call graph only leads to an incorrect \emph{Extract} relationship in the unlikely scenario in which a function $n_2$ is introduced, the content of such function is similar to code removed from $n_1$ and $n_1$ calls a function with the same identifier of $n_2$ after the change, but that function is not actually $n_2$.
A similar reasoning applies to \emph{Inline} relationships, which also depends on information from call graphs.
In summary, although generating precise call graphs is non trivial for untyped languages, we argue that it is not needed in practice to achieve acceptable  precision, specially in the light of the results of our evaluation using JavaScript systems (91\% of precision).

\noindent\textbf{JavaScript class syntax:} Our JavaScript implementation only considers classes defined with the new ES6 syntax, i.e., classes emulated by  functions definitions and prototype-based inheritance are just treated like regular functions when generating the CST.

\noindent\textbf{Field-related refactorings:}
%The current implementation of RefDiff 2.0 does not detect \emph{Move/Pull Up/Push Down Field}, since they are less popular operations~\cite{fse2016-why-we-refactor}.
%However, there is no fundamental issue that prevents supporting these refactorings in the future, as long as we extend the CST to include field information.
As our refactoring detection algorithm is centered around code similarity and fields do not have a body, we did not implement the detection of \emph{Move/Pull Up/Push Down Field} in RefDiff 2.0.
Unrestricted detection of \emph{Move Field} based solely on fields' types and names is prone to find many false positives.
However, we plan to add support for field-related refactorings in future work by using stricter detection rules, similarly to RMiner (e.g., requiring a dependency between their source and destination classes).

%Without resorting to code similarity, we need other means to avoid detecting many false positives. In RefDiff 1.0 we tried to solve that problem by using the statements that used the field in our similarity index. However, such strategy did not yield good results in practice.

\section{Conclusion}
\label{SecConclusion}



%In this paper, we proposed RefDiff 2.0, an extension a refactoring detection approach that is, from the best of our knowledge, 
To the best of our knowledge, RefDiff 2.0 is the first refactoring detection approach that supports multiple programming languages.
%approach designed from the ground up to support multiple programming languages.
We made this possible with two main design decisions. First, our refactoring detection algorithm relies only on information encoded in CSTs, a data structure that represents the source code but abstracts the specificities of each programming language.
Second, we compute code similarity at the level of the tokenized source code, using techniques from information retrieval.
In summary, RefDiff is loosely coupled to the syntax of the target programming language, which makes it easier to extend it to other languages.
Our evaluation using a dataset of real refactorings in Java showed that RefDiff's precision is 96.4\% and recall is 80.4\%.
Although we were not able to surpass RMiner's precision of 98.8\%, we argue that we achieved satisfactory results for a language-neutral approach.
In one hand, specialized tools can use more advanced techniques to improve refactoring detection.
%For example, RMiner relies on a statement matching algorithm and applies syntax-aware replacement of AST nodes to make it more resilient to code changes overlapped with refactorings.
On the other hand, the higher the coupling with the syntax of a particular language, the harder it becomes to port the approach to other programming languages.
Last, our evaluation in JavasScript and C also showed promising results. RefDiff's precision and recall are respectively 91\% and 88\% for JavaScript, and 88\% and 91\% for C.
These results show the viability our approach for languages other than Java.
Thus, we claim that RefDiff~2.0 can pave the way for important advances in refactoring studies in JavaScript, C, and other languages in the future.
Moreover, it can be employed in practical tasks, such as improving diff visualization, automatically documenting refactorings in commits, keeping track of the history of refactored code elements, and others.






% if have a single appendix:
%\appendix[Proof of the Zonklar Equations]
% or
%\appendix  % for no appendix heading
% do not use \section anymore after \appendix, only \section*
% is possibly needed

% use appendices with more than one appendix
% then use \section to start each appendix
% you must declare a \section before using any
% \subsection or using \label (\appendices by itself
% starts a section numbered zero.)
%


%\appendices
%\section{Proof of the First Zonklar Equation}
%Appendix one text goes here.

% you can choose not to have a title for an appendix
% if you want by leaving the argument blank
%\section{}
%Appendix two text goes here.


% use section* for acknowledgment
\ifCLASSOPTIONcompsoc
  % The Computer Society usually uses the plural form
  \section*{Acknowledgments}
\else
  % regular IEEE prefers the singular form
  \section*{Acknowledgment}
\fi


Our research is supported by grants from FAPEMIG and CNPq.


% Can use something like this to put references on a page
% by themselves when using endfloat and the captionsoff option.
\ifCLASSOPTIONcaptionsoff
  \newpage
\fi



% trigger a \newpage just before the given reference
% number - used to balance the columns on the last page
% adjust value as needed - may need to be readjusted if
% the document is modified later
%\IEEEtriggeratref{8}
% The "triggered" command can be changed if desired:
%\IEEEtriggercmd{\enlargethispage{-5in}}

% references section

% can use a bibliography generated by BibTeX as a .bbl file
% BibTeX documentation can be easily obtained at:
% http://mirror.ctan.org/biblio/bibtex/contrib/doc/
% The IEEEtran BibTeX style support page is at:
% http://www.michaelshell.org/tex/ieeetran/bibtex/
\bibliographystyle{IEEEtran}
% argument is your BibTeX string definitions and bibliography database(s)
%\bibliography{IEEEabrv,../bib/paper}
%
% <OR> manually copy in the resultant .bbl file
% set second argument of \begin to the number of references
% (used to reserve space for the reference number labels box)
\bibliography{IEEEabrv,references}

% biography section
% 
% If you have an EPS/PDF photo (graphicx package needed) extra braces are
% needed around the contents of the optional argument to biography to prevent
% the LaTeX parser from getting confused when it sees the complicated
% \includegraphics command within an optional argument. (You could create
% your own custom macro containing the \includegraphics command to make things
% simpler here.)
%\begin{IEEEbiography}[{\includegraphics[width=1in,height=1.25in,clip,keepaspectratio]{mshell}}]{Michael Shell}
% or if you just want to reserve a space for a photo:

%\vskip 0pt plus -1fil
\vskip -2\baselineskip plus -1fil

\begin{IEEEbiography}[{\includegraphics[width=1in,height=1.25in,clip,keepaspectratio]{img-bio-danilofs.jpg}}]{Danilo Silva}
is a Ph.D. candidate in the Computer Science Department at the Federal University of Minas Gerais (UFMG), where he also received a M.Sc. in Computer Science. His research interests include software architecture and modularity, software maintenance and evolution, and refactoring. Contact him at \url{danilofs@dcc.ufmg.br}.
\end{IEEEbiography}

%\newpage
\vskip -2\baselineskip plus -1fil

\begin{IEEEbiography}[{\includegraphics[width=1in,height=1.25in,clip,keepaspectratio]{img-bio-joaopaulo.jpg}}]{João Paulo Ribeiro da Silva}
is a bachelor in Computer Science, majored at the Federal University of Campina Grande (UFCG). He is a developer and team lead at Quimbik, Inc., working on front and back-end software for web systems. His main interest areas are software architecture and modularity, software maintenance and evolution, and web development. Contact him at \url{joao@jpribeiro.com.br}.
\end{IEEEbiography}

\vskip -2\baselineskip plus -1fil

\begin{IEEEbiography}[{\includegraphics[width=1in,height=1.25in,clip,keepaspectratio]{img-bio-gustavo.jpg}}]{Gustavo Santos}
is an assistant professor in Software Engineering at the Federal University of Technology (UTFPR) in Dois Vizinhos, Brazil.
He received his Ph.D. in Informatique from Université de Lille (France) in 2017 and stayed two years at University of São Paulo (USP) for post-doctoral research. 
His research interests include software maintenance and software quality. Contact him at \url{gustavosantos@utfpr.edu.br} or visit \url{gustavojss.github.io}.
\end{IEEEbiography}

\vskip -2\baselineskip plus -1fil

\begin{IEEEbiography}[{\includegraphics[width=1in,height=1.25in,clip,keepaspectratio]{img-bio-terra.jpg}}]{Ricardo Terra}
received his Ph.D. degree in Computer Science from Federal University of Minas Gerais, Brazil (2013) with a 1-year internship at the University of Waterloo, Canada. 
Since 2014, he is an assistant professor in the Department of Computer Science at Federal University of Lavras, Brazil. 
His research interests include software architecture maintainability and evolvability. Contact him at \url{terra@ufla.br}, 
or visit \url{www.dcc.ufla.br/~terra}.
\end{IEEEbiography}

\vskip -2\baselineskip plus -1fil

\begin{IEEEbiography}[{\includegraphics[width=1in,height=1.25in,clip,keepaspectratio]{img-bio-mtov.jpg}}]{Marco Tulio Valente}
is an associate professor in the Computer Science Department at the Federal University of Minas Gerais (UFMG), where he also heads the Applied Software Engineering Research Group (ASERG). His research interests include software architecture and modularity, software maintenance and evolution, and software quality analysis. Valente received a Ph.D. in Computer Science from the Federal University of Minas Gerais. He is a Researcher I-D of the Brazilian National Research Council (CNPq) and holds a Researcher from Minas Gerais State scholarship, from FAPEMIG. Contact him at \url{mtov@dcc.ufmg.br}, or visit \url{www.dcc.ufmg.br/~mtov}.
\end{IEEEbiography}


% insert where needed to balance the two columns on the last page with
% biographies
%\newpage


% You can push biographies down or up by placing
% a \vfill before or after them. The appropriate
% use of \vfill depends on what kind of text is
% on the last page and whether or not the columns
% are being equalized.

%\vfill

% Can be used to pull up biographies so that the bottom of the last one
% is flush with the other column.
%\enlargethispage{-5in}



% that's all folks
\end{document}



